Le second objet d'étude de cette thèse est une technique de détection de logiciels malveillants fonctionnant par comparaison de leurs graphes de flot de contrôle (GFC).

Cette partie est organisée comme suit.
Nous présentons en premier lieu différentes approches de la détection de programmes malveillants, puis nous détaillons l'analyse morphologique fonctionnant par comparaison des graphes de flot de contrôle et expliquons comment optimiser cette approche. Enfin nous donnons une application de l'analyse morphologique à la détection de similarités logicielles et des exemples d'études de codes malveillants basées sur cette approche.

Dans ce chapitre nous présentons plusieurs techniques de détection de programmes malveillants et introduisons l'analyse morphologique.

\section{Contexte de détection}
Nous rappelons la définition d'un programme malveillant, donnée en introduction (définition \ref{def:malware}).

\begin{defi}
Un logiciel est dit malveillant s'il réalise volontairement des opérations allant à l'encontre de l'intérêt de l'utilisateur, et ce à l'insu de celui-ci.
\label{def:malware}
\end{defi}

La définition varie en fait d'un utilisateur à un autre et une technique plus formelle permettant de classifier un programme fait intervenir la spécification d'une politique de sécurité. 
Cette politique de sécurité décrit les différents flots de données autorisés et ceux étant interdits. Par exemple elle pourrait interdire de lire le fichier /etc/shadow contenant des informations relatives aux mots de passe d'un ordinateur sous Linux et d'exfiltrer ces informations hors de la machine (via le réseau ou un support amovible).

Il n'existe pas de programme capable d'analyser tout programme binaire pour connaître a priori leur comportement de manière exacte : on ne peut pas construire un analyseur déterminant si les programmes analysés lisent /etc/shadow.
La raison est que le problème du désassemblage est déjà indécidable ; celui de la détection l'est également.

Puisqu'il n'existe pas de méthode de détection parfaite mais que l'on est en pratique capable d'analyser des programmes à la main, nous disposons d'un corpus de programmes malveillants d'une part et d'un corpus de programmes légitimes d'autre part.
Certaines méthodes mesurent une distance entre le programme analysé et les programmes malveillants connus, puis entre le programme analysé et les programmes légitimes connus. Si le programme analysé est suffisamment proche d'un programme malveillant il sera considéré comme malveillant. S'il est identique ou très proche d'un programme légitime, il pourra être classifié comme légitime.

\paragraph{Programmes obscurcis et \sms.}
La plupart des programmes malveillants utilisent de l'obscurcissement et sont \sms.
De plus certains programmes légitimes sont également \sms, tels ceux permettant la compilation à la volée.
De nombreux éditeurs de logiciels non libres, cherchant à empêcher ou ralentir l'analyse de leur code afin de protéger leur propriété intellectuelle, utilisent le même arsenal de protection que les auteurs de logiciels malveillants.

\paragraph{Critères de classification.}
La classification en logiciel légitime ou malveillant peut s'effectuer sur divers critères, que l'on compare à ceux de programmes ou de comportements connus :
\begin{itemize}
 \item sa représentation sous forme de fichier,
 \item des traces d'exécution,
 \item son code assembleur désassemblé,
 \item le code assembleur du programme dont on aurait enlevé les protections,
 \item son graphe de flot de contrôle (GFC).
%  \item Son graphe de flot (GFC) de contrôle parfait
%  \item Son graphe de flot (GFC) de contrôle paramétré par une exécution
\end{itemize}

Ces caractéristiques et les techniques permettant de les déterminer ont été étudiées dans la première partie de ce document.
Nous considérons que leur extraction ne fait pas à proprement partie de la détection : il s'agit de données à partir desquelles nous pouvons construire des signatures de programmes malveillants ou des comportements de référence.

Dans le cas d'une détection par signatures, les corpus de logiciels malveillants et légitimes contiennent des programmes dont ont préalablement été extraits les critères précédents et à partir desquels des règles de détection ont été décidées.
Le schéma général de détection par signatures est donné en figure \ref{fig:detection}.

\begin{figure}[h]
\begin{center}
\scalebox{1}{
\begin{tikzpicture}[->,scale=1,>=stealth',thick]
\newcommand\espace{0.3cm}
\node[state] (BIN){Binaire};
\node[state, right=3.8cm of BIN.west, anchor=west] (CODE) {Code};
\node[state, above=1.2cm of CODE.west, anchor=west] (TRACE) {Traces};
\node[state, below=1.2cm of CODE.west, anchor=west] (GFC) {GFC};
\coordinate [right=1cm of BIN.east] (DYN);
\coordinate [right=4cm of DYN] (CLASS);
\coordinate [right=1cm of CLASS] (IN);
\coordinate [right=1cm of IN] (OUT);
% \node[state, above=2cm of IN] (COMP) {Liste de comportements};
\node[state, above=2cm of IN] (SIG) {Corpus de programmes classés};
\node[state, above right=1cm and 1cm of OUT, anchor=west] (MAL) {Malveillant};
\node[state, below right=1cm and 1cm of OUT, anchor=west] (LEG) {Légitime};
\draw [-] (BIN.east) -- node[below left=1.6cm and -3.2cm, text width=4cm](DYNAMIC){Extraction} (DYN);
\draw [-] (CLASS) -- (IN);
\draw [-] (IN) -- node[below right=1.6cm and -1.8cm, text width=4cm](STATIC){Classification} (OUT);
\draw (DYN) |- (TRACE.west);
\draw (DYN) |- (CODE.west);
\draw (DYN) |- (GFC.west);
\draw (OUT) -- (MAL.west);
\draw (OUT) -- (LEG.west);
% \draw (COMP.south) -- (IN);
\draw (SIG.south) -- (IN);
\draw [-] (TRACE.east) -| (CLASS);
\draw [-] (CODE.east) -| (CLASS);
\draw [-] (GFC.east) -| (CLASS);
% \draw (In) --  (Vn);
% \node [fit={($(V0.north west) + (-0.2, 0)$) ($(V1) + (0.0, 0)$) ($(Vp) + (0.0, 0)$) ($(Vn.south east) + (0.3, 0)$)}, draw, label=GFC paramétré par la trace] {};
\end{tikzpicture}
}
\end{center}
\caption{Architecture générique d'un détecteur par signatures}
\label{fig:detection}
\end{figure}

Une détection comportementale contient également une liste de comportements connus servant à classifier un programme analysé.


\section{État de l'art}
\paragraph{Approche syntaxique par expressions régulières.}
L'approche traditionnelle de la classification consiste en l'extraction, pour chaque logiciel malveillant du corpus, d'une chaîne de caractères présente dans sa représentation sous forme de fichier \cite{Szor05}.
Elle peut être complétée dans le but de gérer, pour chaque logiciel du corpus, une expression régulière plutôt qu'une chaîne exacte.
Cette approche, purement syntaxique, nécessite qu'un analyste choisisse l'expression régulière à prendre en compte pour chaque logiciel du corpus : il s'agit en général d'une chaîne de données caractéristique (un numéro de version, une clé de chiffrement, un nom, etc.) ou de la représentation en octets de quelques instructions représentatives du programme.
Son avantage est que, mise en \oe uvre correctement, elle génère très peu de faux positifs. Son premier inconvénient majeur est qu'elle est très sensible à une légère modification du programme malveillant : il suffit souvent de rajouter ou de réorganiser le code pour empêcher la détection \cite{CJ04}. Le second inconvénient de cette approche est qu'elle nécessite une analyse manuelle pour extraire la signature, opération d'autant plus contraignante si de nombreuses variantes du programme existent et qu'une nouvelle signature doit être générée pour chaque variante.

\paragraph{Approches comportementales.}
Les méthodes de détection comportementales cherchent à connaître le comportement d'un programme et à le comparer, soit à une liste de comportements connus comme légitimes, soit à des comportements classés comme malveillants \cite{JDF08}.
Il s'agit en général dans un premier temps de définir un certain nombre de comportements de haut niveau, comme l'utilisation de fichiers, de flux réseau ou de certaines fonctionnalités sensibles du système d'exploitation.
Ensuite des scénarios de détection sont définis : il peut s'agir d'un enchaînement d'actions (lecture d'un fichier puis ouverture d'une connexion distante) ou simplement la définition d'un seuil (au delà d'un certain nombre d'actions sensibles effectuées).
La réalisation d'un de ces scénarios par le programme analysé provoque sa classification comme logiciel malveillant.
Certaines implémentations ont montré l'efficacité de l'approche, comme sur l'analyse d'extensions malveillantes pour Internet Explorer \cite{KKBVK06}.

\paragraph{Mesure automatique de distance.}
Une technique générique de détection par signatures consiste à chercher des similarités entre deux programmes en définissant par exemple une distance entre eux. 
Ensuite la classification se fait par mesure de la distance entre le programme analysé et les différents programmes du corpus.
Une possibilité est de décomposer le code des programmes en n-grammes contenant $n$ instructions séquentielles d'un même bloc de base. La signature du programme est alors l'ensemble de ces groupes de $n$ instructions. Pour rendre les instructions plus génériques, Upchurch et Zhou~\cite{UZ13} ne prennent pas les instructions entières mais seulement leur mnémonique (\texttt{mov eax, 3} devient \mov).

\paragraph{Familles de logiciels malveillants.}
Jang, Brumley et Venkataraman \cite{JBV11} proposent un outil générique de classification des logiciels en familles, quelles que soient les caractéristiques et donc les mesures choisies. Ce type d'approches permet d'aller plus loin que la simple classification en logiciel légitime ou malveillant et peut donner des informations sur ce qu'on peut attendre d'un logiciel en fonction de la famille à laquelle il appartient.

\section{Analyse morphologique}
L'analyse morphologique, introduite par Kaczmarek \cite{Kacz08}, Bonfante et Marion \cite{BKM08}, propose une détection 
basée sur la comparaison des graphes de flot de contrôle. L'idée est de comparer la forme des graphes de flot de contrôle plutôt que les instructions exactes qui y sont présentes.
Cette approche fonctionne donc par signatures, les signatures étant des graphes de flot de contrôle, et cherche à mesurer une distance automatique entre un programme et le corpus de programmes malveillants connus.
Prenons le programme assembleur défini à la figure \ref{fig:am_prog}. La structure utile de son graphe de flot de contrôle réside dans les instructions de saut inconditionnel \jne\ aux adresses \adr{8048068} et \adr{8048074}. Ce sont ces deux instructions qui donnent sa forme au GFC : il est constitué de deux chemins principaux dont l'un est constitué d'une boucle.

\begin{figure}[h]
% \def\imagetop#1{\vtop{\null\hbox{#1}}}
\begin{center}
\subfigure[Code assembleur]{
\begin{tabular}[b]{|l|l|l|}
\hline
Adresse & Octets & Instruction\\ 
\hline
 8048060           &  bf 00 00 00 00 &  mov edi, 0 \\
 8048065           &  83 f8 00       &  cmp eax, 0 \\
 8048068           &  75 0e          &  jne 8048078 <suite> \\
		   &		     &			\\
 804806a           &  b8 0a 00 00 00 &  mov eax, 9 \\
 		   &		     &			\\
<boucle>           &		     &			\\
 804806f           &  48             &  dec eax \\
 8048070           &  47             &  inc edi \\
 8048071           &  83 f8 00       &  cmp eax, 0 \\
 8048074           &  75 f9          &  jne 804806f <boucle> \\
 		   &		     &			\\
 8048076           &  eb 01          &  jmp 8048079 <fin> \\
 		   &		     &	            \\
<suite>  	   &		     &			\\
 8048078           &  47             &  inc edi \\
  		   &		     &			\\
<fin>		   &		     &			\\
 8048079           &  c3             &  ret \\
\hline
\end{tabular}
\label{fig:am_prog_asm}
}
\subfigure[Graphe de flot de contrôle]{
\includegraphics[width=0.34\textwidth]{supports/detection/detection_cropped0.pdf}
\label{fig:am_prog_cfg}
}
\end{center}
\caption{Programme et son graphe de flot de contrôle}
\label{fig:am_prog}
\end{figure}


% \FloatBarrier

\subsection{Comparaison des graphes de flot de contrôle}
La technique consiste à rechercher des parties du graphe de flot de contrôle de programmes du corpus dans le programme analysé.
Si une portion suffisante du graphe de flot de contrôle de programmes malveillants connus est présente dans le programme testé, on peut le classer comme indésirable. Formellement il s'agit de trouver des isomorphismes de sous-graphes entre deux graphes de flot de contrôle. Nous définirons plus précisément le problème et détaillerons plusieurs algorithmes de résolution dans le chapitre suivant. Avant de comparer les GFC, ceux-ci sont mis dans une forme canonique à l'aide de réductions : ces réductions réduisent la taille du graphe et le rendent plus générique.

\paragraph{Simplification des sommets.}
Les sommets des GFC sont modifiés afin que chaque instruction soit d'un des types génériques suivants : saut inconditionnel ($JMP$), saut conditionnel ($JCC$), appel de fonction ($CALL$), retour de fonction ($RET$), instruction séquentielle ($INST$) ou instruction inconnue ($UNDEF$). De même la distinction entre les arcs reliant deux instructions séquentielles, en noir dans les GFC, et ceux reliant une instruction de saut et la cible du saut, en rouge, est oubliée.
Cette transformation permet aux signatures d'être plus génériques.

\paragraph{Réduction des graphes de flot de contrôle.}
Plusieurs réductions ont été proposées (figure \ref{fig:am_reductions}) afin de résister à certaines formes d'obscurcissement comme l'insertion de code mort, la réorganisation d'instructions séquentielles, l'ajout de sauts inconditionnels ou d'appels inutiles.

\begin{figure}[h]
\begin{tabular}[t]{|c|c|}
\hline
Concaténation des instructions séquentielles & Réalignement des sauts inconditionnels \\
\hline
\scalebox{0.7}{
\begin{tikzpicture}[->,scale=1,>=stealth',thick, text centered]
\node [draw, circle, text width=1.1cm] (c1) at (0,0) {$I_n$};
\node [draw, circle, text width=1.1cm] (c2) at (0,-2.3) {$INST$};
% \node [left=1cm of c2.west, anchor=east] (LEGENDE) {\large Concaténation des instructions séquentielles};
\node [draw, circle, text width=1.1cm] (c3) at (0,-4.6) {$I_{n+2}$};
\draw[->] (c1) -- (c2);
\draw[->] (c2) -- (c3);

\node [draw, circle, text width=1.1cm] (cp1) at (4,-1.2) {$I_n$};
\node [draw, circle, text width=1.1cm] (cp2) at (4,-3.5) {$I_{n+2}$};
\draw[->] (cp1) -- (cp2);
\coordinate [right=0.7cm of c2.east] (R1);
\coordinate [right=1.6cm of R1] (R2);
\draw[->] (R1) -- node[below=0cm, text width=4cm](RED){Réduction} (R2);
\node [fit={($(c1.north) + (0, 0)$) ($(c1.west) + (0, 0)$) ($(c3.south) + (0.0, 0)$) ($(cp1.east) + (0.0, 0)$)}] {};
\end{tikzpicture}
} &
\scalebox{0.7}{
\begin{tikzpicture}[->,scale=1,>=stealth',thick, text centered]
\node [draw, circle, text width=1.1cm] (c1) at (0,0) {$I$};
\node [draw, circle, text width=1.1cm] (c2) at (0,-2.3) {$JMP$};
% \node [left=1cm of c2.west, anchor=east] (LEGENDE) {\large Réalignement des sauts inconditionnels};
\node [draw, circle, text width=1.1cm] (c3) at (0,-4.6) {$I'$};
\draw[->] (c1) -- (c2);
\draw[->] (c2) -- (c3);

\node [draw, circle, text width=1.1cm] (cp1) at (4,-1.2) {$I$};
\node [draw, circle, text width=1.1cm] (cp2) at (4,-3.5) {$I'$};
\draw[->] (cp1) -- (cp2);
\coordinate [right=0.7cm of c2.east] (R1);
\coordinate [right=1.6cm of R1] (R2);
\draw[->] (R1) -- node[below=0cm, text width=4cm](RED){Réduction} (R2);
\node [fit={($(c1.north) + (0, 0)$) ($(c1.west) + (0, 0)$) ($(c3.south) + (0.0, 0)$) ($(cp1.east) + (0.0, 0)$)}] {};
\end{tikzpicture}
} \\
\hline
\hline
Suppression des appels inutiles & Fusion des sauts conditionnels \\
\hline
\scalebox{0.7}{
\begin{tikzpicture}[->,scale=1,>=stealth',thick, text centered]
\node [draw, circle, text width=1.1cm] (c1) at (0,0) {$I_n$};
\node [draw, circle, text width=1.1cm] (c2) at (0,-2.3) {$CALL$};
% \node [left=1cm of c2.west, anchor=east] (LEGENDE) {\large Suppression des appels (\call) frauduleux};
\node [draw, circle, text width=1.1cm] (c3) at (-1,-4.6) {$I_{n+2}$};
\node [draw, circle, text width=1.1cm] (c4) at (-1,-6.9) {$I_{n+3}$};
\node [draw, circle, text width=1.1cm] (c5) at (1,-4.6) {$RET$};
\draw[->] (c1) -- (c2);
\draw[->] (c2) -- (c3);
\draw[->] (c2) -- (c5);
\draw[->] (c3) -- (c4);

\node [draw, circle, text width=1.1cm] (cp1) at (6,-1.2) {$I_n$};
\node [draw, circle, text width=1.1cm] (cp2) at (6,-3.5) {$I_{n+2}$};
\node [draw, circle, text width=1.1cm] (cp3) at (6,-5.8) {$I_{n+3}$};
\draw[->] (cp1) -- (cp2);
\draw[->] (cp2) -- (cp3);
\coordinate [below right=1cm and 1.3cm of c2.east] (R1);
\coordinate [right=2.6cm of R1] (R2);
\draw[->] (R1) -- node[below=0cm, text width=4cm](RED){Réduction} (R2);
\node [fit={($(c1.north) + (0, 0)$) ($(c3.west) + (0, 0)$) ($(c5.south) + (0.0, 0)$) ($(cp1.east) + (0.0, 0)$)}] {};
\end{tikzpicture}
}&
\scalebox{0.7}{
\begin{tikzpicture}[->,scale=1,>=stealth',thick, text centered]
\node [draw, circle, text width=1.1cm] (c1) at (0,0) {$JCC$};
\node [draw, circle, text width=1.1cm] (c2) at (-1,-2.3) {$JCC$};
% \node [left=1cm of c2.west, anchor=east] (LEGENDE) {\large Fusion des sauts conditionnels};
\node [draw, circle, text width=1.1cm] (c3) at (-1,-4.6) {$I_1$};
\node [draw, circle, text width=1.1cm] (c4) at (1,-4.6) {$I_2$};
\draw[->] (c1) -- (c2);
\draw[->] (c2) -- (c3);
\draw[->] (c2) -- (c4);
\draw[->] (c1) -- (c4);

\node [draw, circle, text width=1.1cm] (cp1) at (7,-1.2) {$JCC$};
\node [draw, circle, text width=1.1cm] (cp2) at (6,-3.5) {$I_1$};
\node [draw, circle, text width=1.1cm] (cp3) at (8,-3.5) {$I_2$};
\draw[->] (cp1) -- (cp2);
\draw[->] (cp1) -- (cp3);
\coordinate [right=2.3cm of c2.east] (R1);
\coordinate [right=2.6cm of R1] (R2);
\draw[->] (R1) -- node[below=0cm, text width=4cm](RED){Réduction} (R2);
\node [fit={($(c1.north) + (0, 0)$) ($(c3.west) + (0, 0)$) ($(c3.south) + (0.0, 0)$) ($(cp3.east) + (0.0, 0)$)}] {};
\end{tikzpicture}
}\\
\hline
\end{tabular}
\caption{Réductions proposées du GFC}
\label{fig:am_reductions}
\end{figure}

Les graphes comparés au final sont les graphes sous forme réduite. Le graphe de flot du programme pris en exemple précédemment est rappelé, simplifié et réduit en figure \ref{fig:am_prog_cfgs}.

\begin{figure}[h]
\begin{center}
\def\imagetop#1{\vtop{\null\hbox{#1}}}
\begin{tabular}[t]{|c|c|c|}
\hline
Graphe de flot de contrôle & GFC simplifié & GFC réduit\\
\hline
\imagetop{\includegraphics[width=0.34\textwidth]{supports/detection/detection_cropped10.pdf}}
&
\imagetop{\includegraphics[width=0.2\textwidth]{supports/detection/detectionSimple_cropped10.pdf}}
&
\imagetop{\includegraphics[width=0.18\textwidth]{supports/detection/detectionReduit_cropped10.pdf}}
\\
\hline
\end{tabular}
\end{center}
\caption{Transformations du graphe de flot de contrôle de l'exemple de la figure \ref{fig:am_prog}}
\label{fig:am_prog_cfgs}
\end{figure}

\paragraph{Résistance à l'obscurcissement.}
L'emploi de réductions a pour but de rendre les signatures plus génériques et moins sensibles à l'obscurcissement.
Certains types d'obscurcissement visent spécifiquement le graphe de flot de contrôle pour le rendre moins intelligible. Nous avons présenté une de ces techniques, l'aplatissement de graphe de flot de contrôle lors de la première partie (section \ref{sec:ex_obsc}). Nous n'avons pas étudié en détail l'impact de ces méthodes sur la précision de la détection faite par l'analyse morphologique mais nous pensons que des techniques d'obscurcissement spécialement créées pour contrer notre technique de détection seraient efficaces et pourraient mener à un grand nombre de faux positifs ou de faux négatifs.

\subsection{Algorithme de comparaison}
L'algorithme de comparaison des graphes de flot de contrôle et de détection d'isomorphismes de sous-graphes développé à l'origine pour l'analyse morphologique \cite{BKM08} transforme les GFC en arbres et crée un automate d'arbres dans lequel les programmes du corpus sont ajoutés.
Le processus de classification d'un programme à analyser découpe son GFC réduit en sous-graphes de petite taille (typiquement entre 12 et 24 sommets), les transforme en arbres et détermine s'ils sont reconnus ou non par l'automate d'arbres.
Un pourcentage de reconnaissance est extrait et selon sa valeur le programme est considéré malveillant ou légitime.

\FloatBarrier
\subsection{Revue de littérature}
Cesare et Xiang \cite{CX10} proposent une approche similaire cherchant également à classer de manière automatique un programme par comparaison de son GFC avec celui des programmes du corpus.
Ils détectent les fonctions assembleur en utilisant des heuristiques et créent un GFC pour chacune de ces fonctions.
Ces GFC sont ensuite transformés dans un langage intermédiaire pour les simplifier, à l'instar des réductions que nous leur appliquons.
% Deux graphes de flot de contrôle de deux fonctions correspondent s'ils sont identiques. 
Une base de GFC est donc créée et les auteurs formalisent une mesure de distance entre un programme et cette base.
La principale différence avec notre approche est donc que les auteurs cherchent à détecter des fonctions dans le programme assembleur tandis que nous construisons un GFC global que nous découpons seulement ensuite.

Zynamics commercialise un outil de comparaison de binaires, BinDiff \cite{bindiff}, qui sépare également les binaires utilisés en fonctions assembleur.
L'outil permet de trouver des fonctions identiques et de comparer des fonctions présentant certaines similarités \cite{DR05}.
Il opère par comparaison de critères sur les fonctions comme leur nom, les appels systèmes qu'elles effectuent ou encore le nombre d'arcs en entrée et en sortie des sommets de leur GFC partiel. Bien qu'il n'utilise pas à proprement parler de techniques d'isomorphismes de graphes, BinDiff fonctionne sur des graphes de flot de contrôle en utilisant certains critères qui leur sont spécifiques et va donc plus loin qu'une simple comparaison syntaxique des fonctions.

L'application de techniques de réduction dans le but de rendre des codes malveillants plus génériques et ainsi que les signatures soient plus efficaces est fréquente dans la littérature. 
Certains travaux \cite{CKJKVM05} proposent des méthodes luttant spécifiquement contre certaines techniques d'obscurcissement comme la réorganisation de code ; c'est l'objectif des réductions proposées à la figure \ref{fig:am_reductions} dont fait partie le réalignement des sauts inconditionnels.
Bruschi, Martignoni et Monga \cite{BMM06} proposent d'utiliser un panel de techniques d'analyse statique existantes pour définir une forme réduite, dite canonique, pour les binaires, puis de comparer les graphes de flots de contrôle résultants.

\section*{Conclusion}
Les approches de la détection des logiciels malveillants fonctionnent par comparaison de signatures ou de comportements.
Les signatures prennent en compte différentes caractéristiques du binaire tandis que les approches comportementales cherchent à obtenir des informations de plus haut niveau sur les exécutions du programme.

L'analyse morphologique, que nous présenterons en détail dans le prochain chapitre, est une approche de la détection des logiciels malveillants fonctionnant par comparaison de signatures.
Les signatures, étant des graphes de flot de contrôle réduits, sont plus génériques que les signatures syntaxiques classiquement utilisées et permettent de détecter des structures similaires dans les différents graphes de flot de contrôle des programmes considérés.