This dissertation tackles the analysis and detection problem of malwares using some protection techniques.

These malwares, unlike regular programs, modify their code during their execution.
We propose an analysis combining execution and passive observation of the program that allows us to recover a graph which represents all the potential actions undertaken by the software.

Moreover we review a malware detector which compares the graphs constructed during the analysis phase. 
This antivirus compares the graph of the studied program against those of known malwares and uses the comparison to classify the program.
We propose a formalization of this graph comparison problem along with an approach to solves it efficiently.