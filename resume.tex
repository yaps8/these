Cette thèse porte en premier lieu sur les programmes malveillants et certaines techniques d'obscurcissement telles que l'auto-modification et le chevauchement de code.
Les programmes malveillants trouvés dans la pratique utilisent massivement l'auto-modification pour cacher leur code utile à un analyste.
Nous proposons une technique d'analyse hybride qui utilise une trace d'exécution déterminée par analyse dynamique.
Cette analyse découpe le programme auto-modifiant en plusieurs sous-parties, non auto-modifiantes, que nous pouvons alors étudier par analyse statique en utilisant la trace comme guide.
Cette seconde analyse contourne d'autres techniques de protection comme le chevauchement de code afin de reconstruire le graphe de flot de contrôle du binaire analysé.

Nous étudions également un détecteur de programmes malveillants, fonctionnant par analyse morphologique : il compare les graphes de flot de contrôles d'un programme à analyser à ceux de programmes connus comme malveillants.
Nous proposons une formalisation de ce problème de comparaison de graphes, des algorithmes permettant de le résoudre efficacement et détaillons des cas concrets d'application à la détection de similarités logicielles.