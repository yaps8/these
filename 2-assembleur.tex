% \newcommand{\xqq}{x862}

Nous nous intéressons en premier lieu aux programmes malveillants fonctionnant sur des ordinateurs personnels.
Les programmes s'exécutant sur ces machines sont compilés afin d'être exécutés nativement dans le langage assembleur spécifique au processeur de la machine.

\section{Compilation et décompilation}
Un programme est généralement écrit dans un langage de haut niveau (C par exemple). Chacun de ses modules est ensuite compilé en un fichier objet (binaire) encodant le langage assembleur spécifique à la machine. La dernière étape est l'édition de liens qui consiste à regrouper tous les fichiers objets en un exécutable unique.

On peut prendre l'exemple d'un simple programme ``Hello World'' en C. Le code source pour un tel programme peut être celui donné en figure ?.

int main(int argc, char* argv[]){
  printf("Hello, world.");
}

\missingfigure{Code C de helloworld}

\begin{lstlisting}[language={C}]
NTSTATUS __cdecl ParsePE(
    return STATUS_WAIT_1; 
\end{lstlisting}


\begin{lstlisting}[language={[x86masm]Assembler}, escapechar=~]
(01) PAGE:004ED1AD                  loc_4ED1AD: [...]                      
(22) PAGE:004ED1EF 3B C3            cmp     eax, ebx
\end{lstlisting}

Le code assembleur \xq\ correspondant, après compilation sous un système d'exploitation Linux (cf syscalls?), est le suivant.
\missingfigure{Assembleur}

Le fichier binaire exécutable résultant de la compilation est le suivant. Il contient des entêtes dans lesquels sont indiqués les différentes sections du programme et deux sections : une section .data contenant les données du programmes (dont la chaîne de caractères ``Hello World'') et une section .text contenant le code assembleur à exécuter.\missingfigure{Binaire}
% \x64
\section{Malwares binaires}
La principale difficulté lors de l'analyse d'un programme malveillant est que le code source n'est pas disponible à l'analyste qui doit se contenter du fichier binaire compilé.

Un programme compilé se présente donc sous la forme d'un fichier binaire contenant le code machine devant être lancé à l'exécution du programme ainsi que des informations de chargement du binaire : la distinctions de différentes sections (sections de code et sections de données), les adresses mémoires auxquelles le système devra les charger en mémoire, etc.

\section{Assembleur \xq\ et \xs}
L'architecture la plus fréquente sur ces ordinateurs personnels est celle des processeurs Intel CISC avec le jeu d'instructions \xq\ pour les machines adressant la mémoire sur 32 bits, et le jeu d'instructions \xs\ pour celles adressant la mémoire sur 64 bits.

\section{Obfuscations}
