Nous cherchons à parcourir le plus de code atteignable et donc avoir une couverture optimale du code lors du désassemblage par rapport à un désassemblage parfait.
Il faut alors pallier à l'incomplétude d'un parcours récursif et au manque de précision d'un parcours linéaire.
Une des difficulté vient de l'obscurcissement par chevauchement de code. 
Nous cherchons donc des approches du désassemblage qui prennent en compte cette méthode de protection.


\section{État de l'art}
% \subsection{Couverture de code}
Bien que le problème du chevauchement de code ne soit pas une technique d'obscurcissement récente et soit bien documenté \cite{PMA}, la littérature portant sur le désassemblage fait souvent l'hypothèse qu'un octet à une adresse spécifique ne peut être présent que dans une seule instruction \cite{KruegelRVV04}. Cette contrainte empêche de détecter tout chevauchement mais permet un désassemblage plus précis sur un binaire qui n'utilise pas cette technique de protection.



Les auteurs de la technique de chevauchement détaillée au chapitre \ref{chap:obscurcissement}, permettant d'encoder une séquence de code cachée dans une séquence \cite{JLH13}, proposent de détecter la protection qu'ils exposent. L'idée est qu'il est improbable qu'une longue séquence d'octets représente une séquence valide de code. Si une telle séquence existe, c'est sûrement du code caché. Cette approche fonctionne pour la protection qu'ils exposent mais n'est pas applicable aux cas de \telock\ ou UPX par exemple car les séquences d'octets sur lesquels des instructions se chevauchent sont très courtes.

\paragraph{Désassembleurs disponibles.}
Les désassembleurs existants, qu'ils utilisent un parcours linéaire ou récursif, font l'hypothèse que le code ne peut pas se chevaucher et ne parviennent pas à afficher un désassemblage cohérent dans le cas contraire.

Le désassemblage récursif de l'exemple de \telock\ avec IDA Pro (version 6.3) \cite{IDA} est le suivant :
\begin{lstlisting}[language={[x86masm]Assembler}, escapechar=~]
01006E7A     inc     byte ptr [ebx+ecx]
01006E7D     jmp     short near ptr loc_1006E7D+1
; ~Les octets qui suivent n'ont pas été désassemblés~
01006E7F     db 0C9h
01006E80     db  7Fh
01006E81     db 0E6h
01006E82     db  8Bh
01006E83     db 0C1h
\end{lstlisting}
Radare \cite{radare} effectue le désassemblage linéaire suivant :
\begin{lstlisting}[language={[x86masm]Assembler}, escapechar=~]
01006e7a    fe 04 0b     inc byte [ebx+ecx]
01006e7d    eb ff        jmp 6e7e
01006e7f    c9           leave
01006e80    7f e6        jg 6e68
01006e82    8b c1        mov eax, ecx
\end{lstlisting}
Ni l'un ni l'autre n'est capable de suivre le saut de l'instruction \jmp\ : la cible du saut a déjà été prise en compte comme faisant partie d'une autre instruction.

De même ni Radare ni IDA ne détectent le second chemin d'exécution dans l'extrait d'UPX et désassemblent cet extrait comme suit.
\begin{lstlisting}[language={[x86masm]Assembler}, escapechar=~]
010059f0    89 f9            mov ecx, edi
010059f2    79 07            jns 0x10059fb
010059f4    0f b7 07         movzx eax, word [edi]
010059f7    47               inc edi
010059f8    50               push eax
010059f9    47               inc edi
010059fa    b9 57 48 f2 ae   mov ecx, 0xaef24857
010059ff    55               push ebp
\end{lstlisting}

\paragraph{Parcours spéculatif.}
Le parcours récursif étant moins sensible à des obscurcissements très simples comme l'injection de code mort, il est souvent pris comme point de départ dans les recherches sur le désassemblage statique.
Une fois une première recherche de code avec un parcours récursif effectuée, les octets restants peuvent subir un parcours linéaire qui cherchera à déterminer s'ils sont du code ou des données à l'aide d'heuristiques.
Une de ces approche évalue la probabilité qu'une suite d'octets soit effectivement du code en apprenant au préalable des suites d'octets codant réellement des instructions lancées lors de l'exécution de programmes \cite{KDF09}.
On appelle cet enchaînement des deux parcours un parcours spéculatif.

Prasaf et Chiueh \cite{PC03}, après un premier désassemblage récursif, tentent d'identifier les adresses de début de fonctions assembleur à l'aide de la suite d'instruction \push\ puis \mov, caractéristique de fonctions compilées : elles commencent par empiler le pointeur de pile de base avec \texttt{push ebp} pour le remplacer par le pointeur de pile avec \texttt{mov ebp, esp}. Ils identifient également toutes les adresses où sont codées une instruction valide. À partir de ces adresses ils effectuent à nouveau un parcours récursif jusqu'à une instruction de saut inconditionnel (\ret\ out \jmp), estimant que la séquence de code identifiée s'arrête sur ce saut. Le risque est grand que des chemins parcourus ainsi ne soient pas valides, ils éliminent alors les chemins qui aboutissent sur du code invalide ou des adresses connues pour être des données.

% \paragraph{Dans la littérature.}

% \\



% \paragraph{Chevauchement de code.}
Krügel, Roberston, Valeur et Vigna \cite{KruegelRVV04} proposent aussi de séparer le code en fonctions assembleur. Ils effectuent une première analyse récursive pour détecter le maximum d'instructions atteignables et, lorsque que deux ou plusieurs instructions atteignables se chevauchent, ils considèrent qu'il s'agit d'un conflit qui se résoudra par le choix d'une des instructions.
Leurs hypothèses sont : (i) les instructions ne peuvent pas se chevaucher, (ii) les sauts conditionnels peuvent être suivis ou non, (iii) le binaire peut contenir du code mort, et (iv) le code suivant une instruction \call\ n'est pas nécessairement accessible. Pour résoudre les conflits de chevauchement ils favorisent les instructions atteignables depuis le point d'entrée (que l'on sait accessible), puis ils considèrent qu'une instruction permettant d'atteindre directement deux instructions qui se chevauchent n'est pas valide et enfin ils favorisent les instructions les plus connectées au graphe de flot de contrôle.
Leur approche est un point de départ pour le désassemble de binaires obscurcis. Ils prennent par exemple en compte l'ajout de code mort et certaines modifications du flot de contrôle.

Schwarz, Debray et Andrews \cite{SDA02} introduisent un parcours linéaire capable de détecter certaines injections de données dans le code, comme les tables de saut (souvent présent dans du code compilé en C ou C++). Ils proposent de combiner les parcours récursifs et linéaires pour détecter les anomalies dans le désassemblage. Si, au sein d'une fonction, une instruction provenant du désassemblage récursif chevauche une instruction provenant du désassemblage linéaire, le désassemblage de la fonction est considéré comme contenant une erreur.
\\

Partant du principe qu'un désassemblage correct ne contient pas de chevauchement de code, ces approches ne sont pas satisfaisantes pour désassembler des binaires utilisant du chevauchement de code, tels les programmes protégés avec \telock.
\\

Dans sa thèse, Kinder \cite{Kinder10} indique que si la technique de désassemblage autorise que différentes instructions désassemblées se chevauchent et réalise des désassemblages atomiques d'une seule instruction à la fois à partir de chaque adresse, alors on peut voir deux instructions qui se chevauchent comme indépendantes. 
Effectivement les travaux présentés dans cet état de l'art ainsi que les désassembleurs existants prennent pour hypothèse que le code ne doit pas se chevaucher alors que nous avons représenté les chevauchements de \telock\ et UPX très simplement. 
L'hypothèse d'alignement permet en général de simplifier les critères de désassemblage et est justifiée par la non occurrence de ce phénomène dans des programmes légitimes.
Nous verrons qu'en pratique l'utilisation du chevauchement de code est rare même dans un binaire protégé et nous proposerons une technique de désassemblage capable de détecter les cas de chevauchement.



\section{Analyse statique et chevauchement de code}
Nous proposons une formalisation du problème du chevauchement de code.
Du point de vue du désassemblage, un programme qui présente une unique instruction qui en chevauche une autre peut se voir comme composé d'un chemin principal de désassemblage et d'un chemin secondaire dans lequel l'instruction en chevauchement se place.
Reprenons l'exemple de \telock : le segment d'octets \texttt{eb ff c9 7f e6} peut se voir comme composé des deux \layers\ de code données à la figure \ref{fig:telock-layers-simple} : il y a deux \layers, la première contient les instructions \texttt{jmp +1}, \texttt{leave} et \texttt{jg 0x1006e68} et la seconde contient l'instruction \texttt{dec ecx}, chevauchant \texttt{jmp +1}.
En fait le segment d'octets \texttt{eb ff c9 7f e6} contient exactement les quatre instructions précédentes : il y a au maximum une instruction valide à chaque adresse et la dernière instruction potentielle, codée sur \texttt{e6}, n'est pas valide.


\begin{figure}
\begin{center}
\begin{tabular}{|l|c|c|c|c|c|}
\hline
Adresses & 0x01006e7d & 0x01006e7e & 0x01006e7f & 0x01006e80 & 0x01006e81\\
\hline
Octets & eb & ff & c9 & 7f & e6\\
\hline
\Layer\ 1 & \multicolumn{2}{c|}{jmp +1} & leave & \multicolumn{2}{c|}{jg 0x1006e68}\\
\hline
\Layer\ 2 & \cnoir & \multicolumn{2}{c|}{dec ecx} & \multicolumn{2}{c|}{\cnoir} \\
 \hline
% \\
\end{tabular}
\end{center}
\caption{Découpage cohérent en \layers\ de l'extrait de \telock}
\label{fig:telock-layers-simple}
\end{figure}

% \begin{figure}
% \begin{center}
% \begin{tabular}{|l|c|c|c|c|c|}
% \hline
% Adresses & 0x01006e7d & 0x01006e7e & 0x01006e7f & 0x01006e80 & 0x01006e81\\
% \hline
% Octets & eb & ff & c9 & 7f & e6\\
% \hline
% \Layer\ 1 & \multicolumn{2}{c|}{jmp +1} & \cnoir & \multicolumn{2}{c|}{jg 0x1006e68}\\
% \hline
% \Layer\ 2 & \cnoir & \multicolumn{2}{c|}{dec ecx} & \multicolumn{2}{c|}{\cnoir} \\
%  \hline
% % \\
% \end{tabular}
% \end{center}
% \caption{\Layers\ de l'extrait de \telock}
% \label{fig:telock-layers-recursive}
% \end{figure}

Formellement on définit une \layer\ comme un ensemble d'instructions qui ne se chevauchent pas (définition \ref{def:layer}).
Par conséquent lors du désassemblage on cherche à effectuer un découpage cohérent des instructions inclues dans le graphe de flot de contrôle en différentes \layers. Un découpage cohérent est simplement la donnée d'un ensemble de \layers\ deux à deux disjointes et recouvrant l'ensemble des instructions désassemblées.
L'exemple précédent pour \telock\ est un découpage cohérent.

\begin{defi}
 Une \layer\ de code $L$ est un ensemble d'instructions dynamiques qui ne se chevauchent pas : $\forall D_1, D_2\in L,\ $\dc{D_1}$\cap$\dc{D_2}$=\emptyset$.
\label{def:layer}
\end{defi}

\paragraph{Borne du nombre de \layers.}
Le nombre minimal de \layers\ permettant de former un découpage cohérent est borné par la taille maximale des instructions, c'est à dire 15 octets pour l'assembleur \xq.
Pour s'en convaincre il suffit de définir le découpage cohérent suivant. Pour un segment d'octets à l'adresse \adr{a}, on place dans la \layer\ i, pour $1\leq i\leq 15$ toutes les instructions aux adresses congrues à $a+i-1 (mod\ 15)$. C'est à dire que la \layer\ 1 contient les instructions aux adresses \adr{a}, \adr{a+15}, \adr{a+30}, etc ; la \layer\ 2 celles aux adresses \adr{a+1}, \adr{a+16}, \adr{a+31}, etc. Puisque les instructions dans chaque \layer\ sont distantes de 15 octets, il ne peut pas y avoir de chevauchement au sein d'une \layer\ et les \layers\ sont disjointes ; il s'agit bien d'un découpage cohérent une fois qu'on enlève les \layers\ ne contenant pas d'instructions valides. Ce découpage contient au plus 15 \layers.
\itodo{l'exemple qui donne 15 layers ?}

Nous définirons par la suite plusieurs découpages cohérents en \layers\ et discuterons de leur pertinence et de leurs implémentations.


\subsection{\Layers\ linéaires}
Une approche naturelle des \layers\ consiste à construire une première \layer\ par parcours linéaire à partir du début de la section de code du binaire.
Cette première \layer\ contient exactement les instructions qu'un désassembleur par parcours linéaire aurait détecté.
Les \layers\ suivantes seront construites également par parcours linéaire, à partir de chaque adresse du binaire.
Ainsi on peut construire une \layer\ de code à partir de chaque adresse du binaire.
Un tel découpage pour \telock\ donne les layers indiqués en figure \ref{fig:telock-layers-linear}.

Ce découpage est donc, par définition, basé sur l'alignement des instructions : au sein d'une \layer, les instructions sont alignées, c'est à dire qu'un désassemblage linéaire depuis la première instruction de la \layer\ parcourt toutes les autres instructions de la \layer. 
Il est à noter qu'en assembleur \xq\ ou \xs\ les instructions ont tendance à se réaligner rapidement : si deux instructions se chevauchent et qu'on l'on réalise un parcours linéaire depuis chacune d'entre elles, il suffit en général d'au plus quatre \todo{cite, vrai?}instructions pour que ces deux parcours se confondent.
Cette propriété a pour conséquence que la plupart des \layers\ linéaires se réalignent sur une \layer\ précédente, en général la première. Sur la figure \ref{fig:telock-layers-linear} les \layers\ 1 et 2 se réalignent et partagent l'instruction \texttt{jg 0x1006e68}.
Cette propriété n'est évidemment pas vérifiée si les instructions ont été spécialement choisie pour provoquer de longs chemins de chevauchement, comme avec l'obscurcissement précédemment cité \cite{JLH13}.

Nous souhaitons obtenir un découpage cohérent des \layers\ sous forme d'ensembles deux à deux disjoints.
Il suffit alors d'enlever des \layers\ les instructions existant déjà dans les \layers\ inférieures (colorées en gris sur la figure \ref{fig:telock-layers-linear}) et de ne garder que les \layers\ contenant des instructions valides.
Au final il reste les deux \layers\ données précédemment à la figure \ref{fig:telock-layers-simple}, obtenus linéairement.

\paragraph{Binaire composé de plusieurs segments.}
On peut étendre la définition précédente applicable à un segment d'octets à un binaire composé de plusieurs segments de code disjoints.
L'extension consiste à simplement à réaliser un découpage cohérent pour chaque segment puis à les réunir au sein d'un seul découpage par une union des \layers\ qui les composent, lui aussi cohérent puisque les segments sont deux à deux disjoints.

\paragraph{\Layers\ linéaires et désassemblage.}
% Les \layers\ linéaires sont en fait une extension du désassemblage linéaire. Un programme non obscurcis peut être désassemblé avec un minimum d'erreurs par un parcours linéaire représenté.
Le découpage en \layers\ linéaires n'est pas en soi un algorithme de désassemblage : il s'agit d'une analyse exhaustive de toutes les instructions codées dans le segment analysé, elle est donc par définition incorrecte.
On peut voir ce découpage comme une caractéristique du binaire qui contient l'ensemble des instructions potentiellement exécutées (hors auto-modification) et qui dépend de leur agencement en mémoire.


\paragraph{Caractérisations des binaires obscurcis.}
Considérons un programme désassemblé en son graphe de flot de contrôle parfait \todo{def}.
Si chaque instruction du graphe de flot de contrôle, présente dans le segment $s$, fait partie de la première \layer\ du découpage linéaire de $s$, il est clair qu'il n'y a pas de chevauchement de code possible.
En pratique c'est le cas la plupart du temps avec des binaires non obscurcis\todo{vrai?}.
L'inverse n'est par contre pas vrai puisque deux instructions peuvent ne pas être alignées à cause d'un ajout de code mort (voir chapitre \ref{chap:obscurcissement}) sans qu'il n'y ait de chevauchement de code dans le graphe de flot.

Lors du désassemblage d'un binaire plusieurs métriques peuvent être observées.
On peut d'une part observer combien de \layers\ différentes sont utilisées : chaque \layer, en dehors de la première de chaque segment, atteste du départ d'un chemin désaligné avec le chemin principal.
D'autre part on peut compter le nombre de sauts d'une instruction d'une \layer\ vers une instruction désalignée d'une autre \layer. Dans le cas de \telock, le saut depuis l'instruction \texttt{dec ecx} à l'adresse \adr{0x01006e7e} vers l'instruction \texttt{jg 0x1006e68} à l'adresse \adr{0x01006e80}, bien qu'il provoque un changement de \layer, n'est pas désaligné. Le nombre de ces sauts de désalignement atteste de la prévalence des chemins représentés par chaque \layer.

\itodo{rapport à un désassemblage linéaire}

\begin{figure}
\begin{center}
\begin{tabular}{|l|c|c|c|c|c|}
\hline
Adresses & 0x01006e7d & 0x01006e7e & 0x01006e7f & 0x01006e80 & 0x01006e81\\
\hline
Octets & eb & ff & c9 & 7f & e6\\
\hline
\Layer\ 1 @0x01006e7d & \multicolumn{2}{c|}{jmp +1} & leave & \multicolumn{2}{|c|}{jg 0x1006e68}\\
\hline
\Layer\ 2 @0x01006e7e & \cnoir & \multicolumn{2}{c|}{dec ecx} & \multicolumn{2}{|c|}{jg 0x1006e68 \cgris} \\
\hline
\Layer\ 3 @0x01006e7f & \multicolumn{2}{c|}{\cnoir} & leave \cgris & \multicolumn{2}{|c|}{jg 0x1006e68 \cgris} \\
\hline
\Layer\ 4 @0x01006e80 & \multicolumn{3}{c|}{\cnoir} & \multicolumn{2}{|c|}{jg 0x1006e68 \cgris} \\
\hline
\Layer\ 5 @0x01006e81 & \multicolumn{4}{|c|}{\cnoir} & (invalide) \\
\hline
% \\
\end{tabular}
\end{center}
\caption{\Layers\ linéaires de l'extrait de \telock}
\label{fig:telock-layers-linear}
\end{figure}

% \begin{figure}
% \begin{center}
% \begin{tabular}{|l|c|c|c|c|c|c|c|c|c|c|}
% \hline
% Addresses & 0xf2 & 0xf3 & ... & 0xf9 & 0xfa & 0xfb & 0xfc & 0xfd & 0xfe & 0xff\\
% \hline
% Bytes & 79 & 07 & ... & 47 & b9 & 57 & 48 & f2 & ae & 55\\
% \hline
% Layer 1 @0xf2 & \multicolumn{2}{c|}{jns +9 (0xfb)} & ... & inc edi & \multicolumn{5}{c|}{mov ecx, aef24857} & push ebp\\
% \hline
% Layer 2 @0xfb & \multicolumn{5}{c|}{\cnoir} & push edi & dec eax & \multicolumn{2}{c|}{repne scasb} & push ebp\\
% \hline
% % \\
% \end{tabular}
% \end{center}
% \caption{Layers of a subset of the UPX code segchevment}
% \label{fig:upx-layers-recursive}
% \end{figure}

\subsection{\Layers\ désassemblées}
Une autre approche consiste à construire des \layers\ de code au fur et à mesure du désassemblage du binaire.
Un désassemblage parfait de l'extrait de \telock\ précédent (figure \ref{fig:telock-layers-simple}) à partir du point d'entrée \adr{0x1006e7d} contient exactement les trois instructions \texttt{jmp +1}, \texttt{dec ecx}, \texttt{jg 0x1006e68}. La première instruction va provoquer la création d'une première \layer, la seconde étant en chevauchement avec la première, elle ne peut être placée que dans une nouvelle \layer. La dernière instruction peut être placée dans n'importe laquelle des deux \layers\ existantes, on la placera, arbitrairement, dans la première \layer. Un tel découpage est donné en figure \ref{fig:telock-layers-rec}.

\begin{figure}[h]
\begin{center}
\begin{tabular}{|l|c|c|c|c|c|}
\hline
Adresses & 0x01006e7d & 0x01006e7e & 0x01006e7f & 0x01006e80 & 0x01006e81\\
\hline
Octets & eb & ff & c9 & 7f & e6\\
\hline
\Layer\ 1 & \multicolumn{2}{c|}{jmp +1} & \cnoir & \multicolumn{2}{c|}{jg 0x1006e68}\\
\hline
\Layer\ 2 & \cnoir & \multicolumn{2}{c|}{dec ecx} & \multicolumn{2}{c|}{\cnoir} \\
 \hline
% \\
\end{tabular}
\end{center}
\caption{Découpage cohérent en \layers\ lors du désassemblage récursif de \telock}
\label{fig:telock-layers-rec}
\end{figure}

% Nous allons définir un algorithme de découpage cohérent en \layers\ lors d'un désassemblage récursif qui sera repris dans notre proposition de désassembleur. Nous tenterons \todo{ahah} de prouver quelques propriétés sur le découpage qui en résulte.

L'algorithme \ref{algo:ajout_inst_layers} explicite le choix fait lors de l'insertion d'une instruction dans un découpage cohérent existant : on choisit simplement la première \layer\ dont les instructions ne chevauchent pas l'instruction à ajouter. La création d'une nouvelle \layer\ peut être nécessaire. Ce découpage étant minimal [à prouver] \todo{à prouver}, il ne peut pas dépasser 15 \layers.

\begin{algorithm}[H] %or another one check
\caption{Ajout d'une instruction à un découpage cohérent en \layers}
\SetAlgoLined
\KwIn{Un découpage cohérent $C$ en $n$ \layers\ $L_i$, une instruction $D$}
\KwResult{Un découpage cohérent contenant l'instruction $D$}
\SetKwProg{Fn}{}{}{}
\SetKwFunction{FRecurs}{ajoutInstruction}
\Fn(
% \tcc*[h]{C : matrice des associations possibles, i : numéro du prochain sommet de P à associer, F : liste des couples d'associations déjà faites}
){\FRecurs{C, D}}{
\eIf {l'instruction D n'est pas comprise dans C} {
  \For {$i$ de $1$ à $n$}{
    \If {$\forall$ instruction $D'\in L_i$, D et D' ne se chevauchent pas}{
      $L_i\leftarrow L_i\cup \{D\}$\\
      \Return $C$
    }
  }
  $L_{n+1}\leftarrow\ \{D\}$ \\
  $C\leftarrow C\cup\{L_{n+1}\}$ \\
  \Return $C$
}
{
  \Return $C$
}
}
\label{algo:ajout_inst_layers}
\end{algorithm}

\paragraph{Non-unicité du découpage pour un parcours récursif.}
La technique de désassemblage que nous proposons est basée sur un parcours récursif. Il est à noter qu'avec ce type de parcours les \layers\ obtenus en applicant l'algorithme précédent dépendent de l'ordre dans lequel les fils de chaque sommet sont explorés.

Reprenons un graphe de flot simplifié de l'échantillon d'UPX donné au chapitre \ref{chap:obscurcissement}, donné en figure \ref{fig:upx_cfg_simple}. Le sommet 1 est le point d'entrée, le sommet 4 le point d'arrêt et les instructions aux sommets 2 et 3 se chevauchent. Le sommet 1 a deux fils : le sommet 2 qui est accessible séquentiellement (les instructions se suivent) et le sommet 3 qui est la cible d'un saut.

Le découpage sera composé de deux \layers\ dans tous les cas mais si l'on parcourt d'abord le sommet 2 alors les deux \layers\ sont $L_1=\{1, 2, 4\}$ et $L_2=\{3\}$. Au contraire si l'on choisit de d'abord parcourir le sommet 3, les \layers\ sont $L_1=\{1, 3, 4\}$ et $L_2=\{2\}$.

\begin{figure}
\begin{center}
\includegraphics[width=0.2\textwidth]{supports/disasm/upx/upx_simple.pdf}
\end{center}
\caption{Graphe de flot de contrôle simplifié de l'échantillon d'UPX}
\label{fig:upx_cfg_simple}
\end{figure}

\paragraph{Métriques.}
Le découpage précis en \layers\ dépend de la manière dont le parcours récursif est effectué mais \todo{à prouver} le nombre de \layers\ ainsi que le nombre de sauts de changement de \layers\ n'en dépendent pas.
On utilisera le nombre de \layers\ pour observer la complexité des chevauchements de code tandis que le nombre de sauts de changement de \layers\ indique la fréquence d'utilisation de cette technique d'obscurcissement.


\itodo{Découpage cohérent minimal.}
\itodo{discussion : pourquoi on préfère celui là ?}