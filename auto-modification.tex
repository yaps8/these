Nous avons jusqu'à présent fait l'hypothèse que les programmes analysés ne sont pas auto-modifiants.
Dans ce chapitre nous détaillons une approche d'analyse dynamique adaptée aux programmes auto-modifiants et ayant pour but de réduire l'analyse d'un programme auto-modifiant en l'analyse de plusieurs sous-ensembles non auto-modifiants du programme.
Cette approche est donnée par la littérature existante, en particulier par les travaux de thèse de \nom{Reynaud} \cite{Reynaud2010} et \nom{Calvet} \cite{Calvet2013}.

L'analyse dynamique consiste à se baser sur une ou des exécutions particulières d'un programme pour inférer des propriétés sur son fonctionnement.
Elle demande donc de se munir d'un modèle pour le fonctionnement de la machine durant l'exécution du programme, d'une sémantique concrète de l'assembleur utilisé et de lancer l'évaluation sémantique du programme sur une ou plusieurs entrées : nous reprendrons à cet effet le langage assembleur et la sémantique concrète définis au chapitre précédent comme référence.
% Les entrées d'un programme sont par exemple les paramètres passés lors de l'appel au programme, l'état de la machine lors du démarrage, les données lues depuis la mémoire et les saisies faites par l'utilisateur au clavier ou à la souris lors de l'exécution s'il s'agit d'une application dotée d'une interface graphique.
% Nous reprendrons la sémantique simplifiée pour un langage assembleur définie au chapitre précédent et expliquerons comment séparer une exécution d'un programme auto-modifiant en plusieurs sous-exécution non auto-modifiantes afin d'utiliser des techniques standard d'analyse sur chacune de ces sous-exécutions.

\section{Auto-modification et vagues}
Prenons le programme auto-modifiant donné à la figure \ref{fig:prg_asm_sm} et analysons une exécution de ses instructions 1 à 5.
Deux instructions provoquent une auto-modification.
L'instruction 2 à l'adresse \adr{8048065} va provoquer une écriture à l'adresse pointée par la valeur de \eax\ incrémentée de 1, soit à l'adresse \adr{8048077}. Cette écriture modifie l'instruction 4 en \texttt{mov edi, 2}, codée sur les octets \texttt{bf 02 00 00 00}. De même l'instruction 3 modifie l'instruction 5 en \texttt{mov ebx, 2}, codée sur \texttt{bb 02 00 00 00}.
 
\begin{figure}[h]
\begin{center}
\begin{tabular}[b]{|l|l|l|l|}
\hline
i & Adresse & Octets & Instruction\\ 
\hline
1 & 8048060  &  b8 6b 80 04 08         &  mov    eax, 0x8048076\\
2 & 8048065  &  66 c7 40 01 02 00      &  mov    [eax+1], 2 \\
3 & 804806b  &  66 c7 40 06 02 00      &  mov    [eax+6], 2 \\
4 & 8048076  &  bf 01 00 00 00         &  mov    edi, 1 \\
5 & 804807b  &  bb 01 00 00 00         &  mov    ebx, 1 \\
\hline
\end{tabular}
\end{center}
\caption{Programme auto-modifiant}
\label{fig:prg_asm_sm}
\end{figure}

Au vu de l'enchaînement des instructions, on peut construire trois représentations en mémoire des parties exécutables du programme.
La première correspond à la vision du programme lors de son chargement : la section \ptext\ est dans son état initial, donné en figure \ref{fig:prg_asm_sm}.
La seconde est celle après que la première modification du programme faite par l'instruction 2 et la troisième après la seconde modification effectuée par l'instruction 3.
En fait vu qu'aucune des adresses modifiées par la première instruction auto-modifiante n'est exécutée avant que la seconde modification ne soit faite, on peut regrouper les deux instructions auto-modifiantes et considérer que le programme n'a que deux représentations en mémoire : la représentation initiale et la représentation après que l'instruction 3 ait été exécutée. 
Beaucoup de logiciels protégés restaurent de grandes parties de code au démarrage : un grand nombre d'instructions sont modifiées sans que les instructions modifiées ne soient exécutées. 
Il nous paraît plus cohérent que toutes ces modifications soient réunies puisqu'elles correspondent à une étape lors de l'exécution du programme, l'étape suivante étant l'exécution d'une de ces instructions modifiées.

Dans ce découpage informel on appelle vague le désassemblage d'un instantané de la mémoire, pris à un instant donné. 
L'exécution d'un programme est alors caractérisée par une suite d'exécutions sur des vagues successives comme représenté en figure \ref{fig:vagues_visuel}. Les instructions qui sont présentes dans la trace sont colorées en rose tandis que le point d'entrée et la dernière instruction sont en orange et bleu clair, respectivement.
On passe d'une vague $k$ à la vague suivante $k+1$ lorsqu'une adresse mémoire écrite dans la vague $k$ est exécutée.
Ainsi dans une vague $k$, toutes les instructions exécutées ont été écrites au moins à la vague $k-1$. 
En ce sens chacune des vagues, prise indépendamment des autres, ne présente pas d'auto-modification.

Nous détaillerons par la suite, formellement, ce qu'est une trace d'exécution pour l'analyse dynamique ainsi que la sémantique d'enchaînement des vagues.

\begin{figure}
 \includegraphics[width=1.0\textwidth]{supports/automodification/phases2_final_test.pdf}
 \vspace{0.2cm}
 \caption{Vision informelle des vagues}
 \label{fig:vagues_visuel}
\end{figure}
% 
\section{Trace, niveaux d'exécution et vagues}
Le programme exécuté a pour sources principales de données les registres et la mémoire constituée de la pile et du tas qui sont tous les deux adressables par des entiers. Une variable d'un programme est donc soit un registre du processeur soit une adresse mémoire, de même que défini dans la sémantique du langage intermédiaire défini au chapitre précédent (définition \ref{def:sem_conc_var}).

En utilisant la sémantique concrète précédemment définie on est capable, à partir d'un ensemble de valeurs initiales pour les registres et la mémoire, d'exécuter un programme sur cette entrée.
Schématiquement, l'exécution d'un programme consiste en la définition d'un contexte d'exécution initial $E_0$, l'évaluation sémantique de la première instruction $D_1$ dans ce contexte, puis l'évaluation de l'instruction suivante dans le contexte mis à jour, et ainsi de suite. Dans cette partie nous définirons formellement ces éléments mais leur enchaînement est schématiquement le suivant.

% \begin{figure}
\begin{center}
\scalebox{1}{
\begin{tikzpicture}[->,scale=1,>=stealth',thick]
\node[state, draw=none] (E0){$E_0$};
\node[state, draw=none, right=2cm of E0] (E1){$E_1$};
\node[state, draw=none, right=2cm of E1] (E2){$E_2$};
\node[state, draw=none, right=2cm of E2] (EP){$...$};
\node[state, draw=none, right=2cm of EP] (EN){$E_n$};
\draw (E0.east) -> node[above]{$D_1$} (E1.west);
\draw (E1.east) -> node[above]{$D_2$} (E2.west);
\draw (E2.east) -> node[above]{...} (EP.west);
\draw (EP.east) -> node[above]{$D_n$} (EN.west);
\end{tikzpicture}
}
\end{center}
% \caption{Enchaînement des contextes d'exécution et des instructions dynamiques}
% \label{fig:mem_process}
% \end{figure}

Nous rappelons que nous notons, à tout instant durant l'exécution, $\BT$ le tableau contenant les adresses mémoires et $\Theta$ le store représentant l'état des variables (mémoire et registres), comme indiqué aux définitions \ref{def:sem_conc_var} et \ref{def:sem_store_dynamique}.
Nous définissons une instruction dynamique (définition \ref{def:ensembles_inst_dyn}) par son adresse, les adresses mémoires sur lesquelles elle est codée et l'instruction machine correspondant. Ces informations sont données par le décodage d'une instruction à l'adresse mémoire spécifiée dans un contexte donné. 

\begin{defi}
On note $D$ une instruction dynamique constituée des éléments suivants.
\begin{itemize}
 \item \da{D} l'adresse mémoire de l'instruction dynamique
 \item \dc{D} l'intervalle des adresses mémoire sur lequel $D$ est codée
 \item \di{D} l'instruction machine à l'adresse \da{D}
%  \item \dr{D_i} l'ensemble des variables sur lesquelles l'exécution de \di{D_i} provoque une lecture
%  \item \dw{D} l'ensemble des variables sur lesquelles l'exécution de \di{D} provoque une écriture
\end{itemize}
On définit l'opérateur \texttt{decode} qui associe à une adresse mémoire $a$ et un store $\Theta$ l'instruction dynamique $D=$\texttt{decode}$(a, \Theta)$ présente à l'adresse $a$.
\label{def:ensembles_inst_dyn}
\end{defi}

Si l'on dispose d'un langage intermédiaire comme défini au chapitre précédent, son opérateur de désassemblage atomique nous donne ces informations. Pour l'assembleur \xq, une instruction est codée au maximum sur 15 octets.
% Si on cherche à désassembler \telock\ à l'adresse \adr{01006e7a}, et qu'à l'adresse \adr{01006e7d} sont présents les octets suivant : $\Theta[0x01006e7d..0x01006e8b]=$ \texttt{eb ff c9 7f e6 8b c1 29 00 00 00 f3 aa 66 ab}, l'opérateur \texttt{decode} renvoie la première instruction dynamique D à cette adresse : il s'agit de l'instruction assembleur \di{D}=\texttt{jmp +1} (d'une taille de deux octets : \texttt{eb ff}) à l'adresse \da{D}=\adr{01006e7d}, codée sur l'intervalle d'adresses \dc{D}=[\adr{01006e7d}, \adr{01006e7e}].
On cherche à désassembler l'exemple précédent à l'adresse \adr{8048060}. À l'adresse \adr{8048060} sont présents les octets suivants : $\Theta[$\adr{8048060}..\adr{804806e}$]=$ \texttt{b8 6b 80 04 08 b8 6b 80 04 08 66 c7 40 06 02}, l'opérateur \texttt{decode} renvoie la première instruction dynamique D à cette adresse : il s'agit de l'instruction assembleur \di{D}=\texttt{mov eax, 0x8048076} (d'une taille de cinq octets : \texttt{b8 6b 80 04 08}) à l'adresse \da{D}=\adr{8048060}, codée sur l'intervalle d'adresses \dc{D}=[\adr{8048060}..\adr{8048064}].

Afin de pouvoir séparer la trace d'exécution selon le moment où chaque instruction a été écrite, nous définissons aussi, pour une instruction dynamique et un store $\Theta$, l'ensemble des adresses mémoire sur lesquelles l'instruction provoque une écriture (définition \ref{def:ensembles_inst_dyn_write}).

\begin{defi}
Soit $D$ une instruction dynamique et $\Theta$ un store représentant l'état des variables (mémoire et registres). On note \dww{\Theta}{D} l'ensemble des adresses mémoire sur lesquelles l'exécution de $D$ provoque une écriture.
\label{def:ensembles_inst_dyn_write}
\end{defi}

Cette information est donnée par la sémantique concrète choisie : si l'instruction assembleur \di{D} est sémantiquement équivalente, avec le store $\Theta$, à la suite d'instructions atomiques $d_1, ..., d_n$ dans un langage intermédiaire alors l'ensemble des adresses écrites par D est l'union de adresses écrites par chacune des instructions atomiques $d_i$ dans la sémantique concrète choisie pour ce langage intermédiaire (propriété \ref{propri:eq_W_D_di}).

\begin{propri}
Soit $D$ une instruction dynamique, $\Theta$ un store représentant l'état des variables (mémoire et registres) et $d_1, ..., d_n$ une suite d'instruction atomiques dans un langage intermédiaire telle que la suite d'instruction $d_1, ..., d_n$ dans le langage intermédiaire est sémantiquement équivalente à l'instruction assembleur \di{D}. On a :
$$\mdww{\Theta}{D}=\mdww{\Theta}{d_1,..., d_n}=\mdww{\Theta}{d_1}\ \cup\ ...\ \cup\ \mdww{\Theta}{d_n}.$$
\label{propri:eq_W_D_di}
\end{propri}

Avec la sémantique définie au chapitre précédent les seules instructions qui provoquent des écritures sont celles dont la liste d'instructions atomiques donnée par le désassemblage contiennent des assignations écrivant, par adressage direct ou indirect, à une adresse mémoire $m$. Si \di{D} est une instruction assembleur sémantiquement équivalente, avec le store $\Theta$, à l'enchaînement des instructions atomiques $d_1, ..., d_n$ dans le langage intermédiaire précédent, alors \dww{\Theta}{D}=\dww{\Theta}{d_1}$\ \cup\ ...\ \cup\ $\dww{\Theta}{d_n} (propriété \ref{propri:eq_W_D_di}) avec
% \\
$$\mdww{\Theta}{d_i}=
\left\{
  \begin{array}{ll}
	  m &$ si $d_i=m\leftarrow g(x_1, ..., x_k)\ et\ m\in\BT
	\\\Theta(v) &$ si $d_i=[v]\leftarrow g(x_1, ..., x_k)\ et\ \Theta(v)\in\BT
	\\ \emptyset &$ sinon.$
  \end{array}
\right.
$$

% \begin{defi}
% Nous définissons le niveau d'écriture d'une adresse mémoire comme un entier naturel et le store $W^M: \BT\rightarrow\BN$ associant à une adresse mémoire son niveau d'écriture.
% \label{def:store_mem}
% \end{defi}

Nous définissons plus formellement la notion de contexte d'exécution (définition \ref{def:contexte_exec}) comme la donnée d'un niveau d'exécution, d'un store contenant les valeurs de la mémoire et des registres et d'un store contenant les niveaux d'écriture courants de chaque adresse mémoire.

\begin{defi}
Un contexte d'exécution est la donnée d'un triplet $E=(X, \Theta, W)$ où
\begin{itemize}
 \item $X\in\BN$ est le niveau d'exécution du contexte
 \item $\Theta$ est le store des valeurs du contexte, associant une valeur à chaque registre et chaque adresse mémoire
 \item $W$ est le store des niveaux d'écriture du contexte, associant à chaque adresse mémoire un niveau d'écriture dans $\BN$
\end{itemize}
\label{def:contexte_exec}
\end{defi}

Une exécution (définition \ref{def:execution}) consiste en une série de contextes d'exécution dont la transition de l'un au suivant est provoquée par l'évaluation sémantique de l'instruction pointée par le registre de compteur ordinal (noté en général \pc\ ou \eip\ en assembleur \xq). 

\begin{defi}
Une exécution d'un programme, dont le point d'entrée est $ep$, est la donnée d'une suite finie de contextes d'exécution $E_0, E_1, ..., E_n$ tel que :
\begin{itemize}
 \item $X_0=1$, $\Theta_0[eip]=ep$ et $\forall m\in \BT, W_0[m\leftarrow 0]$
 \item En notant $D_{i+1}=$\texttt{decode}$(\Theta_i[eip], \Theta_i)$ l'instruction dynamique exécutée lors de la transition entre le contexte $E_i$ et $E_{i+1}$, on a :
    \begin{itemize}
     \item Le niveau d'écriture de $D_{i+1}$ est le niveau d'écriture maximum des octets qui la composent : $W_D=max(\{W[m],\ m\in\ $\dc{D_{i+1}}$\})$
     \item $X_{i+1}=max(X_i, W_D+1)$
     \item $\Theta_{i+1}$ est $\Theta_i$ mis à jour par l'évaluation sémantique de \di{D}
     \item $W_{i+1}=W_{i}$ sauf pour les adresses mémoire écrites par $D_{i+1}$:\\ $\forall m\in\ $\dww{\Theta}{D_{i+1}}$,\ W_{i+1}[m\leftarrow X_i]$
    \end{itemize}
\end{itemize}
\label{def:execution}
\end{defi}

Une trace d'exécution est une signature d'une exécution, elle contient les instructions successives avec leurs niveau d'exécution respectifs (définition \ref{def:trace}).

\begin{defi}
Étant donnée une exécution d'un programme composée des contextes d'exécution $E_0, E_1, ..., E_n$ avec $E_i=(X_i, \Theta_i, W_i)$, on appelle trace d'exécution la suite $T=(t_1, t_2, ..., t_n)$ où $t_i=(i, X_{i-1}, D_i)$ avec $D_i=$\texttt{decode}$(\Theta_{i-1}[eip], \Theta_{i-1})$.
\label{def:trace}
\end{defi}

\sloppy{Nous avons donc, pour une exécution donnée, des niveaux d'exécution successifs $1, 2, ..., n$.
À chaque contexte d'exécution toute adresse en mémoire $m$ a un niveau d'écriture $W[m]$ correspondant au dernier niveau d'exécution durant lequel une instruction a modifié la valeur à l'adresse $m$.
Une instruction $D$ a un niveau d'écriture $W_D$ qui est le niveau d'écriture le plus élevé parmi les adresses sur lesquelles elle est codée.
Lors d'une exécution le niveau d'exécution ainsi que le niveau d'écriture de chaque adresse mémoire sont croissants.}



% \begin{defi}
% Nous définissons une trace d'exécution comme la donnée d'une suite $T=t_1, t_2, ..., t_n$ composée de triplets de la forme $t_i=(i, D_i, X_i)$ tels que
% \begin{itemize}
%  \item $D_i$ est la $i^{eme}$ instruction dynamique exécutée.
%  \item Avant l'exécution de l'instruction $D_i$, le niveau d'exécution est \texttt{$X_{i-1}$}.
%  \item Après l'exécution de l'instruction $D_i$, le niveau d'exécution est \texttt{$X_i$}.
% \end{itemize}
% \label{def:write_exec_levels}
% \end{defi}



% \begin{propri}
%  Si le niveau d'exécution courant est $X$, le niveau d'exécution de l'instruction à exécuter $D_i$ est :\\
%  $X=max(X, W_D+1)$ avec $W_D=max(W^M[a],\ a\in\ $\dc{D_i}$)$.\\
%  Après l'exécution de $D_i$, les niveaux d'écriture dans la mémoire sont mis à jour de la manière suivante :\\
%  $\forall a\in$ \dw{D_i}, $W^M[a]=X$.
% \label{propri:niveau_exec}
% \end{propri}

En pratique une instruction $D$ écrite par une instruction ayant pour niveau d'exécution $k$ puis directement exécutée aura pour niveau d'écriture $W_D=k$ et pour niveau d'exécution $X=k+1$. On définit alors un instantané du niveau d'exécution $k$ selon la définition \ref{def:instantane} et une vague comme étant le graphe de flot de contrôle parfait de cet instantané (définition \ref{def:vagues}).

\begin{defi}
 Étant donnée une exécution d'un programme composée des contextes d'exécution $E_0, E_1, ..., E_n$ avec $E_i=(X_i, \Theta_i, W_i)$, on appelle instantané du niveau d'exécution $k$ l'état de la mémoire contenu dans $\Theta_j$ où $E_j$ est le premier contexte d'exécution dont le niveau d'exécution est $X_j=k$.
 On appelle point d'entrée et point de sortie de cet instantané respectivement $D_{in}=$\texttt{decode}$(\Theta_{j}[eip], \Theta_{j})$ et  $D_{out}=$\texttt{decode}$(\Theta_{l}[eip], \Theta_{l})$ où $E_l$ est le dernier contexte d'exécution dont le niveau d'exécution est $X_l=k$.
 \label{def:instantane}
\end{defi}

\begin{defi}
 Étant donnée une exécution d'un programme, on appelle vague $k$ le graphe de flot de contrôle parfait du programme représenté par l'instantané du niveau d'exécution $k$ muni de son point d'entrée et considéré comme \nsm.
 \label{def:vagues}
\end{defi}

Une première vague est définie dès l'exécution de la première instruction du programme puis à chaque changement de niveau d'exécution une nouvelle vague est construite.
L'algorithme \ref{algo:analyse_dyn_vagues} (utilisant les algorithmes \ref{algo:update_exec_level} et \ref{algo:update_write_level}) permet d'exécuter un programme dynamiquement avec la sémantique concrète choisie tout en déterminant les niveaux d'exécution et d'écriture au fur et à mesure de l'exécution. 
Il fournit en sortie la trace d'exécution et la liste des instantanés d'exécution reconstruits.

\begin{algorithm}[h] %or another one check
\caption{Mise à jour du niveau d'exécution d'une instruction}
\SetAlgoLined
\KwIn{La mémoire, l'opérateur de niveau d'écriture, une instruction dynamique et le niveau d'exécution courant}
\KwResult{Le niveau d'exécution courant mis à jour}
\SetKwProg{Fn}{}{}{}
\SetKwFunction{FRecurs}{MAJExecution}
\Fn(
){\FRecurs{M, $W$, D, X}}{
$W_D \leftarrow\ max(W[a],\ a\in\ $\dc{D}$)$\\
$X \leftarrow\ max(X,\ W_D+1)$ \\
\Return X
}
\label{algo:update_exec_level}
\end{algorithm}

\begin{algorithm}[h] %or another one check
\caption{Mise à jour des niveaux d'écriture lors de l'exécution d'une instruction}
\SetAlgoLined
\KwIn{La mémoire, l'opérateur de niveau d'écriture, une instruction dynamique et le niveau d'exécution courant}
\KwResult{L'opérateur de niveau d'écriture mis à jour}
\SetKwProg{Fn}{}{}{}
\SetKwFunction{FRecurs}{MAJEcriture}
\Fn(
){\FRecurs{M, $W$, D, X}}{
\For {$m\in\ $\dw{D}}{
  $W[m]\leftarrow\ X$
}
\Return $W$
}
\label{algo:update_write_level}
\end{algorithm}

\begin{algorithm}[h] %or another one check
\caption{Analyse dynamique avec calcul des vagues}
\SetAlgoLined
\KwIn{Les registres R et une mémoire M dans laquelle un programme de point d'entrée \texttt{ep} a été chargé}
\KwResult{La trace des instructions dynamiques chacune associée à leur niveau d'exécution et les différentes vagues de la trace}
\SetKwProg{Fn}{}{}{}
\SetKwFunction{FRecurs}{analyseDynamique}
\Fn(
% \tcc*[h]{C : matrice des associations possibles, i : numéro du prochain sommet de P à associer, F : liste des couples d'associations déjà faites}
){\FRecurs{R, M, ep}}{
\For{$m\in M$}{
  $W[m]\leftarrow 0$\\
}
$(X, X_{-1}, i, T, instantanes, eip)\leftarrow (1, 0, 1, \emptyset, \emptyset, ep)$\\
% $X\leftarrow 1$\\
% $X_{-1}\leftarrow 0$\\
% $i\leftarrow 0$\\
% $T\leftarrow \emptyset$\\
% $vagues\leftarrow \emptyset$\\
% $eip\leftarrow ep$\\
\While {la fin du programme n'est pas atteinte}{
\tcc{Le programme est exécuté en prenant l'instruction suivante à l'adresse eip}
$D\leftarrow decode(eip, M)$\\
$X\leftarrow MAJExecution(M, W, D, X)$\\
\If {$X \ne X_{-1}$}{
  \tcc{Quand le niveau d'exécution change, on prend un instantané de la mémoire}
  $instantanes\leftarrow instantanes\cup \{(X_{-1}, M)\}$
}
$X_{-1} \leftarrow X$\\
~\\
\tcc{On met à jour le contexte à partir de l'instruction courante en l'évaluant sémantiquement}
$(eip, R, M)\leftarrow sem\_eval(eip, R, M)$\\
$W\leftarrow MAJEcriture(M, W, D, X)$\\
$T\leftarrow T\cup\{(i, X, D_i)\}$\\
$i\leftarrow i+1$\\
}
\Return T, instantanes
}
\label{algo:analyse_dyn_vagues}
\end{algorithm}

Reprenons l'exemple du programme \sm\ précédent.
La figure \ref{fig:prg_asm_sm_trace} donne une trace d'exécution de ce programme en détaillant les informations sur chaque instruction dynamique ainsi que les niveaux d'écriture et d'exécution de chaque instruction.
Au départ toute la mémoire est dans son état d'origine et a pour niveau d'exécution 0. Lorsque l'instruction $D_1$ est exécutée, il n'y a pas eu d'auto-modification donc le niveau d'écriture est 0 et le niveau d'exécution est 1.
Les instructions $D_2$ et $D_3$ provoquent une auto-modification : les octets aux adresses \adr{8048077} et \adr{804807c} sont modifiés et leurs niveaux d'écriture deviennent donc le niveau d'exécution courant, soit 1.
Lorsque l'exécution atteint $D_4$, qui a été modifié, le niveau d'écriture est 1 donc le niveau d'exécution devient 2.
L'instruction suivante $D_5$ a également un niveau d'écriture de 1 donc le niveau d'exécution est inchangé.

\begin{figure}[h]
\begin{center}
\begin{tabular}[b]{|l|l|l|l|l|l|l|}
\hline
i & \da{D_i} & \dc{D_i} & \di{D_i} & \dw{D_i} & $W^\Theta_i$ & $X_i$ \\
\hline
1 & 8048060  & [8048060, 8048064] & mov    eax, 0x8048076  &           & 0 & 1 \\
2 & 8048065  & [8048065, 804806a] & mov    [eax+1], 2      & 0x8048077 & 0 & 1 \\
3 & 804806b  & [804806b, 8048075] & mov    [eax+6], 2      & 0x804807c & 0 & 1 \\
4 & 8048076  & [8048076, 804807a] & mov    edi, 1          &           & 1 & 2 \\
5 & 804807b  & [804807b, 804807f] & mov    ebx, 1          &           & 1 & 2 \\
\hline
\end{tabular}
\end{center}
\caption{Trace d'exécution du programme auto-modifiant de la figure \ref{fig:prg_asm_sm}}
\label{fig:prg_asm_sm_trace}
\end{figure}

Cette exécution est donc séparée en deux vagues : la vague initiale, $v_0$ dont l'instantané est l'état de la mémoire avant l'exécution de la première instruction et la vague $v_1$ contenant l'état de la mémoire juste après l'exécution de $D_3$ et avant l'exécution de la première instruction modifiée $D_4$.


% \begin{figure}
% \begin{center}
% \begin{tabular}[b]{|l|l|l|l|l|l|l|}
% \hline
% i & \da{D_i} & \dc{D_i} & \di{D_i} & \dw{D_i} & $W_i$ & $X_i$ \\
% \hline
% & 8048060  &  (...)         	        & Pile -> RWX &  & 0 & 1 \\ 
% 1 & 804807c  &  [804807c, 8048080]         &  mov    edi, 0x0 & edi & 0 & 1 \\
% 2 & 8048081  &  [8048081, 8048086]         &  mov    eax, 0x8048091 & eax & 0 & 1 \\
% 3 & 8048086  &  [8048086, 804808a]         &  mov    [eax], 0xeb & 0x8048091 & 0 & 1 \\
% 4 & 804808b  &  [804808b, 8048090]         &  mov    [eax+1], 0x7 & 0x8048092 & 0 & 1 \\
% 5 & 8048091  &  [8048091, 8048092]         &  jmp    80480a1 <edi3> &  & 1 & 2  \\
% 6 & 804809a  &  [804809a, 804809d]         &  mov    edi,0x2 & edi & 0 & 2\\
% 7 & 804809f  &  [804809f, 80480a0]         &  jmp    80480a8 <fin> &  & 0 & 2\\
%  & 80480a8  &  (...)		        &  Affiche edi &  & 0 & 2\\
%  & 80480c3  &  (...)		        &  Quitte &  & 0 & 2\\
% \hline
% \end{tabular}
% \end{center}
% \caption{Trace d'exécution du programme auto-modifiant de la figure \ref{fig:unevague_v0}}
% \label{fig:unevague_trace}
% \end{figure}

\paragraph{Non unicité des instructions dynamiques dans la trace.}
 Étant donné la définition croissante des niveaux d'exécution, une instruction dynamique peut-être exécutée non seulement plusieurs fois au même niveau d'exécution mais également être présente à des niveaux d'exécution différents.

\paragraph{Exécution d'un programme avec une entrée.}
Supposons que l'on dispose d'un programme P dont le point d'entrée est $ep$ et prenant une entrée.
Nous appelons exécution de ce programme sur une entrée I composée d'états initiaux pour les registres $I_R$ et pour la mémoire $I_M$, sous réserve que le programme P est correctement chargé dans $I_M$, la donnée de la trace et des instantanés d'exécution résultant de l'application de l'algorithme \ref{algo:analyse_dyn_vagues}, soit \texttt{analyseDynamique($I_R$, $I_M$, ep)} que l'on notera par la suite \texttt{exécution(P, I, ep)}.

\section{Revue de littérature}
La notion de vague présentée dans ce chapitre a été développée dans les thèse de Reynaud \cite{Reynaud2010} et Calvet \cite{Calvet2013}.
Elle est similaire à la notion de \emph{phase} présentée par Debray et Patel \cite{DP10} et utilisée pour automatiser la suppression de la protection d'un binaire. Le découpage d'une trace en phases est, chez eux, identique au découpage en vagues que l'on présente dans ce chapitre.
En particulier le programme auto-modifiant précédent (figure \ref{fig:prg_asm_sm}), exécuté, forme deux vagues : la vague initiale est le programme initial et la seconde, modifiée, est identique à l'exception des instructions 4 et 5 qui ont été modifiées en \texttt{mov edi, 2} et \texttt{mov ebx, 2} respectivement.
En général la suppression des protections se fait à l'aide d'une analyse dynamique et d'une image de la mémoire à un instant donné au cours de l'exécution. C'est cette image mémoire qui sera considérée comme étant le programme d'origine. La difficulté réside alors dans le choix de l'instant où prendre l'image mémoire : il s'agit souvent la dernière phase.

Preda, Giacobazzi, Debray, Coogan et Townsend \cite{PGDCT10} effectuent un découpage en phases mais chaque exécution d'une instruction écrite lors d'une vague précédente (pas uniquement lors de la phase en cours) provoque la création d'une nouvelle phase.
Avec l'exemple précédent, leur approche crée trois phases : la phase initiale, celle où uniquement l'instruction 4 a été modifiée, puis celle où les instructions 4 et 5 ont été modifiées.

\section{Implémentations}
Plusieurs choix s'offrent à qui cherche à implémenter un système d'analyse dynamique de binaire tels l'émulation, l'instrumentation et le débogage.

L'émulation consiste à lancer l'exécution dans un environnement d'exécution simulé, qui peut être un système d'exploitation complet comme c'est le cas avec BAP ou TEMU, le module d'analyse dynamique du projet BitBlaze \cite{bitblaze08}, basé sur l'émulateur QEMU \cite{QEMU05}.

On instrumente un binaire exécuté en y insérant, généralement au cours de son exécution, du code assembleur servant à son analyse. Intel développe PinTools \cite{pintools} pour l'analyse de programmes tournant sur leurs processeurs.

Enfin le débogage suit pas à pas l'exécution d'un programme en utilisant le drapeau de trappe (\emph{Trap Flag}), permettant de reprendre la main après chaque instruction du programme débogué afin d'examiner son environnement d'exécution.
\\

Le débogage comme l'instrumentation n'utilisent pas de langage intermédiaire tandis qu'un émulateur tel que BAP transcrit d'abord les instructions dans son langage intermédiaire pour les exécuter avec la sémantique concrète du langage intermédiaire.
L'émulation est donc intéressante parce qu'à aucun moment le programme analysé n'a un accès libre au système sur lequel il s'exécute.
La limitation est donc que les interactions du programme émulé avec le système d'exploitation visé sont restreintes.
En particulier les appels systèmes, qui ne sont pas transcrits par BAP, ne peuvent pas être émulés directement, rendant l'analyse très partielle.
Les approches nécessitant une exécution non restreinte sur le système sont alors réalisées au sein d'une machine virtuelle.

Une caractéristique cruciale d'un analyseur dynamique est qu'il doit être transparent : le programme analysé ne doit pas être capable de différencier son exécution dans l'analyseur de son exécution sur un système réel.
Cette transparence est en général partielle, que ce soit avec un émulateur, un débogueur ou une technique d'instrumentation \cite{Ferrie07}.\more{cite blue pill ou autre}

L'instrumentation, par rapport au débogage, offre des performances temporelles d'exécution supérieures.
Pour ces raisons, nous nous sommes intéressés à l'émulation comme à l'instrumentation.
\\

L'émulation permet une analyse plus abstraite et nous avons développé un analyseur partiel de programmes auto-modifiants basé sur BAP.
L'instrumentation permet d'exécuter plus fidèlement le programme à analyser, nous avons donc principalement favorisé cette approche pour l'analyse de programmes malveillants. Nous avons choisi Pin qui, sans fournir de sémantique concrète pour l'assembleur, permet d'obtenir d'une instruction dynamique l'ensemble des adresses sur lesquelles elle écrit, comme souhaité à la définition \ref{def:ensembles_inst_dyn_write}.

\subsection{Émulation avec BAP}
Nous avons repris l'implémentation d'un émulateur pour programmes \sms\ basée sur BAP présentée au chapitre \ref{chap:semantique}. Nous avions modifié BAP afin de pouvoir exécuter des instructions une à une et avons ainsi réalisé un émulateur de programmes auto-modifiants, fonctionnalité qui n'est pas supportée telle quelle par BAP.

Nous y avons ajouté la séparation de la trace en plusieurs vagues d'exécution en implémentant l'algorithme \ref{algo:analyse_dyn_vagues}.
Cette implémentation donne le résultat attendu avec des programmes simples et qui n'utilisent pas d'appels système.
Sur l'exemple de programme \sm\ donné en début de chapitre, la sortie est la suivante.
L'exécution, de la première (\emph{pc: 1}) à la dernière instruction, est correctement découpée en vagues et les instructions correspondent à celles modifiées lors de l'exécution.

\begin{center}
\begin{verbatim}
Vague 1, pc: 1: addr 0x8048060 @asm "mov    eax, 0x8048071"
Vague 1, pc: 2: addr 0x8048065 @asm "movw   [eax+1], 2"
Vague 1, pc: 3: addr 0x804806b @asm "movw   [eax+6], 2"
Vague 2, pc: 4: addr 0x8048071 @asm "mov    edi, 2"
Vague 2, pc: 5: addr 0x8048076 @asm "mov    ebx, 2"
\end{verbatim}
\end{center}

Ne pouvant pas émuler des programmes utilisant des appels systèmes, nous avons laissé de coté cette approche pour une approche par instrumentation permettant l'analyse de programmes réels.


\subsection{Instrumentation avec Pin}
Pin est l'outil d'instrumentation développé par Intel pour ses processeurs \xq\ et \xs.
Il fournit une information sémantique de niveau 2, suffisante pour suivre chaque instruction et connaître l'ensemble des adresses mémoire que l'instruction modifie.
Nous avons implémenté l'algorithme \ref{algo:analyse_dyn_vagues} sous la forme d'un PinTool : codé en C, il s'agit d'un programme définissant les actions à effectuer avant et après l'exécution de chaque instruction selon le type de l'instruction considérée.
% \itodo{Attention TraceSurfer=daniel mais il n'y avait pas l'algo, c'est arrivé après}

L'analyse d'un programme malveillant a lieu sur une machine virtuelle munie de Pin et ne souffre pas des limitations de l'émulation : les appels systèmes sont correctement analysés et exécutés. 
Dans la suite de cette thèse, nous avons exclusivement utilisé cette approche pour l'analyse dynamique.

\section*{Conclusion}
Afin d'analyser les programmes \sms, nous proposons de découper toute exécution en une trace et une suite d'instantanés d'exécution : chaque instantané représente une partie \nsm e du programme.
Nous avons implémenté ce découpage par émulation avec BAP et par instrumentation avec Pin.

Le chapitre suivant est consacré à l'analyse statique du problème de chevauchement de code.
Le chapitre \ref{chap:desassembleur} portera sur l'analyse d'un programme \sm\ : nous combinerons les notions abordées dans ce chapitre et le chapitre suivant afin de reconstruire les vagues correspondant à chaque niveau d'exécution ainsi qu'un graphe de flot de contrôle global pour le binaire étudié.
