This dissertation explores tactics for analysis and disassembly of malwares using some obfuscation techniques such as self-modification and code overlapping.
Most malwares found in the wild use self-modification in order to hide their payload from an analyst.
We propose an hybrid analysis which uses an execution trace derived from a dynamic analysis.
This analysis cuts the self-modifying binary into several non self-modifying parts that we can examine through a static analysis using the trace as a guide.
This second analysis circumvents more protection techniques such as code overlapping in order to recover the control flow graph of the studied binary.

Moreover we review a morphological malware detector which compares the control flow graph of the studied binary against those of known malwares.
We provide a formalization of this graph comparison problem along with efficient algorithms that solve it and a use case in the software similarity field.