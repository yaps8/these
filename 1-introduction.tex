% Ceci est une introduction.\todo{Mettre une intro}
% \todo[inline]{Une bonne introduction.}
\todo[inline]{Contexte, enjeux\\ Buts de la thèse\\ Contributions}

% \section{Contexte et enjeux}
\todo[inline]{Contexte, enjeux:}
\todo[inline]{Malware, sécurité, détection}
% Un logiciel opérant sur un ordinateur est dit malveillant s'il réalise volontairement des opérations allant à l'encontre de l'intérêt de l'utilisateur et ce à l'insu de celui-ci. 
Un logiciel malveillant réalise volontairement des opérations allant à l'encontre de l'intérêt de l'utilisateur. 
Certaines de ces actions peuvent être objectivement malveillantes comme c'est le cas avec Stuxnet qui vise des éléments d'un système de contrôle industriel afin de le rendre inutilisable. 
% Les administrateurs utilisent quotidiennement des programmes de gestion de machine à distance. 
Un logiciel malveillant visant à fournir à un attaquant un accès à distance ne diffère pas fondamentalement, en termes de fonctionnalités, d'un logiciel légitime utilisé par un administrateur.
La différence entre les deux est que l'administrateur n'est pas au courant de l'installation du logiciel malveillant et ne peut pas le contrôler.
Nous proposons donc la définition suivante pour un logiciel malveillant.
\begin{defi}
Un logiciel est dit malveillant s'il réalise volontairement des opérations allant à l'encontre de l'intérêt de l'utilisateur, et ce à l'insu de celui-ci.
\end{defi}

Les actions malveillantes peuvent être de trois nature. On parle d'atteinte à la confidentialité lorsque des données privées sont obtenues par l'attaquant, d'atteinte à l'intégrité lorsque de l'information présente sur le système attaqué est modifiée par l'attaquant et d'atteinte à la disponibilité du système si l'attaquant rend le système inutilisable ou le surcharge\todo{dispo?}.

Les logiciels malveillants s'attaquent en général à la confidentialité et la disponibilité du système bien que les outils d'administration à distance soient capables, une fois la machine infectée, d'attaquer les trois aspects de la sécurité de la machine.




\section{Détection de logiciels malveillants}
\todo[inline]{+ anti-détection}

\section{Organisation du document}