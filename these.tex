%             Exemple d'utilisation de la classe thesul
%             ------------------------------------------
%
%
% (de manière generale, les commandes de thesul sont celles
% qui ne sont pas complètement en minuscules)
%
% Voir la documentation complète pour plus de détails.
%   
%
% D. Roegel, 30 mars 2013
%
% \documentclass[12pt, oneside]{TUL/thesul}
\documentclass[12pt]{TUL/thesul}
%----------------------------------------------------------------------
%                     Chargement de quelques packages
%----------------------------------------------------------------------

% Si l'on veut produire une version PDF avec des hyperliens :
% \usepackage[pageanchor=false]{TUL/tulhypref}
% \usepackage[hidelinks, pdftex]{TUL/tulhypref}
\usepackage{etex}
\usepackage[hidelinks]{TUL/tulhypref}
% \usepackage[sc]{mathpazo}
\linespread{1.0}
% Si on veut le style de bibliographie named :
%\usepackage{named}

% Pour les figures :
\usepackage{graphicx}

% Si on veut des mini-tables des matières (utiliser minitoc-hyper 
% en conjonction avec tulhypref) :
\usepackage[french]{minitoc}

\usepackage{titlesec}
\usepackage{url}
\usepackage{listings}
\usepackage{pstricks}
\usepackage{subfigure}
\usepackage{amsmath}
\usepackage{amsthm}
\usepackage{amssymb}
\usepackage{tabularx}
\usepackage{textcomp}
\usepackage{multirow}
\usepackage[algoruled,french,onelanguage,algochapter]{algorithm2e}
\usepackage[section]{placeins}
\usepackage{xcolor}
\usepackage{colortbl}
\usepackage{longtable}

\usepackage{epstopdf}
\usepackage{graphicx} % pour insérer des images
\usepackage{stmaryrd}
\usepackage{amsfonts}
% \usepackage{tikz}
% \usepackage{tikz-qtree}
% les figure imbriquées
\usepackage{epsfig}
% \usepackage{enumerate}
\usepackage{pifont}

\usepackage{pgf}
\usepackage{tikz, calc}
% \usepackage{tikz-cd}
\usetikzlibrary{positioning}
\usetikzlibrary{fit}
\usetikzlibrary{shapes.multipart,calc}
\usetikzlibrary{arrows}

% \usepackage[math]{iwona} 
% \usepackage{iwona} 
% \SetMathAlphabet{\mathtt}{iwona}{OT1}{\ttdefault}{m}{n}

\usepackage[backend=biber, language=french, maxnames=10, citestyle=alphabetic,bibstyle=alphabetic,backref,abbreviate=false,dateabbrev=false,isbn=false,url=false,doi=true]{biblatex}
% \usepackage[backend=biber, language=french, maxnames=5,backref,abbreviate=false,dateabbrev=false,isbn=false,url=false,doi=true]{biblatex}
\addbibresource{these.bib}
% \usepackage[font=small,skip=0pt]{caption}
\usepackage[skip=0pt]{caption}
\usepackage{etoolbox}
\usepackage{needspace}

% Todo
% \newcommand{\done}[1]{}
% \newcommand{\idone}[1]{}
\newcommand{\itodo}[1]{\todo[inline]{#1}}

\newcommand{\done}[1]{\todo[color=green!80!blue!80]{#1}}
\newcommand{\idone}[1]{\todo[inline,color=green!80!blue!80]{#1}}

\newcommand{\jym}[1]{\todo[color=red!80!blue!50]{#1}}
\newcommand{\ijym}[1]{\todo[inline,color=red!80!blue!50]{#1}}

\newcommand{\more}[1]{\todo[color=purple!60!blue!30]{#1}}
\newcommand{\imore}[1]{\todo[inline,color=purple!60!blue!30]{#1}}

% Samples
\newcommand{\telock}{tElock}

% Color cells
\newcommand{\cnoir}{\cellcolor[gray]{0.0}}
\newcommand{\cgris}{\cellcolor[gray]{0.8}}

% Traductions
\newcommand{\Layer}{Couche}
\newcommand{\Layers}{Couches}
\newcommand{\layer}{couche}
\newcommand{\layers}{couches}
% \newcommand{\Layer}{Strate}
% \newcommand{\Layers}{Strates}
% \newcommand{\layer}{strate}
% \newcommand{\layers}{strates}

% sm
\newcommand{\sm}{auto-modifiant}
\newcommand{\nsm}{non auto-modifiant}

% Structure
\newtheorem{pb}{Problème}
\newtheorem{theo}{Théorème}
\newtheorem{defi}{Définition}[chapter]
\newtheorem{prop}{Proposition}
\newtheorem{propri}{Propriété}
\newtheorem{pr}{Preuve}
\newtheorem{cor}{Corollaire}
\newtheorem{rem}{Remarque}
% \theoremstyle{remark}\newtheorem*{preuve}{Preuve}


% % Abbreviations :
\newcommand{\helloworld}{\texttt{Hello World}}
\newcommand{\nasm}{NASM}
\newcommand{\xq}{x86}
\newcommand{\xs}{x86$\_$64}
\newcommand{\pdata}{\texttt{.data}}
\newcommand{\ptext}{\texttt{.text}}

% Adresses mémoire :
\newcommand{\adr}[1]{$#1$}

% Registres
\newcommand{\eax}{\texttt{eax}}
\newcommand{\ebx}{\texttt{ebx}}
\newcommand{\ecx}{\texttt{ecx}}
\newcommand{\edx}{\texttt{edx}}
\newcommand{\edi}{\texttt{edi}}
\newcommand{\eip}{\texttt{eip}}
\newcommand{\esp}{\texttt{esp}}

% Instructions
\newcommand{\mov}{\texttt{mov}}
\newcommand{\cmp}{\texttt{cmp}}
\newcommand{\add}{\texttt{add}}
\newcommand{\jmp}{\texttt{jmp}}
\newcommand{\ret}{\texttt{ret}}
\newcommand{\call}{\texttt{call}}
\newcommand{\push}{\texttt{push}}
\newcommand{\jne}{\texttt{jne}}
\newcommand{\je}{\texttt{je}}
\newcommand{\sub}{\texttt{sub}}
\newcommand{\dec}{\texttt{dec}}
\newcommand{\nop}{\texttt{nop}}
\newcommand{\halt}{\texttt{halt}}
\newcommand{\pc}{\texttt{pc}}

% Sémantique statique
\newcommand{\BN}{\mathbb N}
\newcommand{\BE}{\mathbb E}
\newcommand{\BV}{\mathbb V}
\newcommand{\BX}{\mathbb X}
\newcommand{\BA}{\mathbb A}
\newcommand{\BP}{\mathbb P}
\newcommand{\BT}{\mathbb T}
\newcommand{\BL}{\mathbb L}
\newcommand{\BB}{\mathbb B}
\newcommand{\BNB}{\mathbb N\cup\{\bot\}}
\newcommand{\PN}{\mathcal{P}(\BN)}
\newcommand{\PMN}{\mathcal{P}_M(\BN)}
\newcommand{\Trs}{\mathbb N\cup\{\bot,\top\}}
\newcommand{\TTrs}{\mathcal P(\mathbb V\rightarrow\mathbb N\cup\{\bot,\top\})}
\newcommand{\Tr}{\PN\cup{\top,\bot}}
\newcommand{\TrM}{\PMN\cup\{\top,\bot\}}
\newcommand{\si}{\sigma_{init}}
\newcommand{\specialcell}[2][c]{%
  \begin{tabular}[#1]{@{}l@{}}#2\end{tabular}}
  
% Sémantique dynamique
\newcommand{\CA}{\mathcal A}
\newcommand{\CI}{\mathcal I}
\newcommand{\CC}{\mathcal C}
\newcommand{\CR}{\mathcal R}
\newcommand{\CW}{\mathcal W}
\newcommand{\da}[1]{$\CA[#1]$}
\newcommand{\di}[1]{$\CI[#1]$}
\newcommand{\dc}[1]{$\CC[#1]$}
\newcommand{\dr}[1]{$\CR[#1]$}
\newcommand{\dw}[1]{$\CW[#1]$}
\newcommand{\dww}[2]{$\CW^{#1}[#2]$}

% tikz
\tikzset{
  state/.style={
    rectangle,
    rounded corners=1pt,
    draw=black, very thick,
    minimum height=2em,
    text centered,
  },
}

% itemize
% \renewcommand{\labelitemii}{$\cdot$}
% \renewcommand{\labelitemi}{$\bullet$}
% \renewcommand{\labelitemii}{$\cdot$}
% \renewcommand{\labelitemiii}{$\diamond$}
% \renewcommand{\labelitemiv}{$\ast$}

\DefineBibliographyStrings{french}{
        in = {},%
%       in = {\emph{Dans}}%
        backrefpage = {cité page},
        backrefpages = {cité pages}
}

\titleformat{\chapter}[display]
  {\bfseries\huge}
  {\filleft\Large\chaptertitlename~\thechapter}
  {1ex}
  {\titlerule\vspace{1.5ex}\filright}
  [\vspace{1ex}\titlerule]

%-------------------------------------------------------------------
%                             Marges
%-------------------------------------------------------------------

% pour positionner les vraies marges:
%\SetRealMargins{1mm}{1mm}

%-------------------------------------------------------------------
%                             En-têtes
%-------------------------------------------------------------------

% Les en-têtes: quelques exemples
%\UppercaseHeadings 
%\UnderlineHeadings
%\newcommand\bfheadings[1]{{\bf #1}}
%\FormatHeadingsWith{\bfheadings}
%\FormatHeadingsWith{\uppercase}
%\FormatHeadingsWith{\underline}
\newcommand\upun[1]{\uppercase{\underline{\underline{#1}}}}
\FormatHeadingsWith\upun

\newcommand\itheadings[1]{\textit{#1}}
\FormatHeadingsWith{\itheadings}

% pour avoir un trait sous l'en-tete:
\setlength{\HeadRuleWidth}{0.4pt}

\usepackage[backgroundcolor=blue!20!white, linecolor=black]{todonotes}

%-------------------------------------------------------------------
%                         Les références
%-------------------------------------------------------------------

j\NoChapterNumberInRef
\NoChapterPrefix

%-------------------------------------------------------------------
%                           Brouillons
%-------------------------------------------------------------------

% ceci ajoute une marque « brouillon » et la date
%\ThesisDraft

%-------------------------------------------------------------------
%                   Pour collecter un glossaire et un index
%-------------------------------------------------------------------

\makeglossary
\makeindex


\begin{document}
\OddHead={{\leftmark\rightmark}{\hfil\slshape\rightmark}}
\EvenHead={{\leftmark}{{\slshape\leftmark}\hfil}}
\OddFoot={\hfil\thepage}
\EvenFoot={\thepage\hfil}
\pagestyle{ThesisHeadingsII}

%-------------------------------------------------------------------
%                          Encadrements
%-------------------------------------------------------------------

% encadre les chapitres dans la table des matières:
% (ces commandes doivent figurer apres \begin{document}

\FrameChaptersInToc  
%\FramePartsInToc


%-------------------------------------------------------------------
%            Réinitialisation de la numérotation des chapitres
%-------------------------------------------------------------------

% Si la commande suivante est présente,
% elle doit figurer APRÈS \begin{document}
% et avant la première commande \part
% \ResetChaptersAtParts 

%-------------------------------------------------------------------
%               mini-tables des matières par chapitre
%-------------------------------------------------------------------

% préparer les mini-tables des matières par chapitre.
% (commande de minitoc.sty)
\dominitoc

%-------------------------------------------------------------------
%                         Page de titre:
%-------------------------------------------------------------------

\ThesisTitle{Analyse morphologique de logiciels malveillants auto-modifiants}
\ThesisDate{dd/mm/aaaa}
\ThesisAuthor{Aurélien Thierry}

% Type de la these
\ThesisUL

% Jury:

% (ne pas mettre de \\ apres la dernière entree)

% Exemple de création d'une nouvelle catégorie dans le jury:

% \NewJuryCategory{family}{\it Membre de la famille :}
%                         {\it Membres de la famille :}
% 
% \family={Mon frère\\Ma sœur}

\def\blanc{\hspace*{1cm}}

\President    = {Le président}
\Rapporteurs  = {Le rapporteur 1&de Paris\\
                 Le rapporteur 2\\
                 \blanc suite&taratata\\
                 Le rapporteur 3}
\Examinateurs = {L'examinateur 1&d'ici\\
                 L'examinateur 2}
%\Invites=       {}

% Création de la page de titre:
\MakeThesisTitlePage

% on peut en faire plusieurs:
%\MakeThesisTitlePage

%-------------------------------------------------------------------


%-------------------------------------------------------------------
%                          remerciements
%-------------------------------------------------------------------

%\DontFrameThisInToc
\begin{ThesisAcknowledgments}
Merci.
\end{ThesisAcknowledgments}

%-------------------------------------------------------------------
%                            dédicace
%-------------------------------------------------------------------

% \begin{ThesisDedication}
% Ah.
% \end{ThesisDedication}


%-------------------------------------------------------------------
%                  écriture de `Chapitre' et `Partie' 
%                      dans la table des matières
%-------------------------------------------------------------------

\WritePartLabelInToc
\WriteChapterLabelInToc

%-------------------------------------------------------------------
%                        table des matières
%-------------------------------------------------------------------

\tableofcontents

%-------------------------------------------------------------------
%              Exemple d'utilisation de \SpecialSection
%-------------------------------------------------------------------

% \SpecialSection{Introduction générale}


% Pour ne pas avoir le mot « Chapitre » au début de chaque chapitre.
% \NoChapterHead

\DontWriteThisInToc   
\listoffigures


% La commande \mainmatter permet de passer
% à la numérotation arabe (ce que fait \pagenumbering{arabic}) 
% et de faire commencer la nouvelle page 1 sur une page impaire.
% On évitera donc d'utiliser directement \pagenumbering{arabic}.
\mainmatter

\DontFrameThisInToc
\NumberThisInToc
\chapter{Introduction}
% Ceci est une introduction.\todo{Mettre une intro}
% \todo[inline]{Une bonne introduction.}
\todo[inline]{Contexte, enjeux\\ Buts de la thèse\\ Contributions}

% \section{Contexte et enjeux}
\todo[inline]{Contexte, enjeux:}
\todo[inline]{Malware, sécurité, détection}
% Un logiciel opérant sur un ordinateur est dit malveillant s'il réalise volontairement des opérations allant à l'encontre de l'intérêt de l'utilisateur et ce à l'insu de celui-ci. 
Un logiciel malveillant réalise volontairement des opérations allant à l'encontre de l'intérêt de l'utilisateur. 
Certaines de ces actions peuvent être objectivement malveillantes comme c'est le cas avec Stuxnet qui vise des éléments d'un système de contrôle industriel afin de le rendre inutilisable. 
% Les administrateurs utilisent quotidiennement des programmes de gestion de machine à distance. 
Un logiciel malveillant visant à fournir à un attaquant un accès à distance ne diffère pas fondamentalement, en termes de fonctionnalités, d'un logiciel légitime utilisé par un administrateur.
La différence entre les deux est que l'administrateur n'est pas au courant de l'installation du logiciel malveillant et ne peut pas le contrôler.
Nous proposons donc la définition suivante pour un logiciel malveillant.
\begin{defi}
Un logiciel est dit malveillant s'il réalise volontairement des opérations allant à l'encontre de l'intérêt de l'utilisateur, et ce à l'insu de celui-ci.
\end{defi}

Les actions malveillantes peuvent être de trois nature. On parle d'atteinte à la confidentialité lorsque des données privées sont obtenues par l'attaquant, d'atteinte à l'intégrité lorsque de l'information présente sur le système attaqué est modifiée par l'attaquant et d'atteinte à la disponibilité du système si l'attaquant rend le système inutilisable ou le surcharge\todo{dispo?}.

Les logiciels malveillants s'attaquent en général à la confidentialité et la disponibilité du système bien que les outils d'administration à distance soient capables, une fois la machine infectée, d'attaquer les trois aspects de la sécurité de la machine.

Ainsi, en général, un logiciel malveillant diffère d'un logiciel légitime en ce qu'il cherche à dissimuler son existence et son action sur le système. Il déploie à cette fin des techniques de protection logicielles rendant son analyse plus difficile que celle du programme légitime.


\section{Détection de logiciels malveillants}
\todo[inline]{+ anti-détection}
Les premiers travaux traitant de virologie informatique datent de 1986 avec Cohen \cite{Cohen86} puis Adleman \cite{Adleman88} en 1988. Ils s'intéressent particulièrement au comportement auto-reproducteur de certains programmes. Adleman \todo{vrai?} aboutit au résultat suivant dans le cas des logiciels malveillants : la classification d'un logiciel comme malveillant est un problème $\Pi_2$ complet, soit plus difficile que le problème de l'arrêt qui est déjà indécidable.

Les décennies qui ont suivi ont vu apparaître différentes techniques de détection partielles dont la plus répandue est l'analyse de signatures syntaxiques. Les approches par signatures consistent dans un premier temps à extraire d'un corpus de logiciels malveillants connus des caractéristiques comme certaines instructions ou plus généralement une expression rationnelle. Dans un second temps, pour classifier un logiciel inconnu, on regarde s'il possèdent une des caractéristiques extraites du corpus.
Cette technique, dans le cas des expressions rationnelles, possède le double avantage d'être rapide et de générer peu de fausses alarmes.

Cependant chaque souche originale d'un logiciel malveillant est généralement déclinée en de nombreuses versions dont les fonctionnalités varient. Ces souches formes une famille de logiciels malveillants. Pour éviter la détection par signatures syntaxiques, il suffit souvent d'insérer du code inutile ou de réorganiser le code.

Nous nous sommes alors intéressés à la technique de détection initiée par Kaczmarek \cite{AThierry_BKM08} lors de sa thèse. Il s'agit d'une technique de détection par signatures basée sur la comparaison de graphes de flots de contrôles, c'est à dire du graphe structurant l'exécution du logiciel.
\todo[inline]{exemple prog -> CFG}

Dans la pratique, les logiciels malveillants sont souvent protégés par une technique d'encapsulation cachant complètement la charge utile à un analyste 

\section{Problèmes de recherche}


\section{Organisation du document}

% \WriteThisInToc
% \FrameThisInToc
% \NumberThisInToc
\part{Désassemblage et analyse de binaires}

% \NumberThisInToc
\DontFrameThisInToc
\chapter{Assembleur}
Nous nous intéressons en premier lieu aux programmes malveillants fonctionnant sur des ordinateurs personnels.
Les programmes s'exécutant sur ces machines sont compilés afin d'être exécutés nativement dans le langage assembleur spécifique au processeur de la machine.


\section{Malwares binaires}

\section{Assembleur x86 et x86$\_$64}
L'architecture la plus fréquente sur ces ordinateurs personnels est celle des processeurs Intel CISC

\section{Obfuscations}


\DontFrameThisInToc
\chapter{Techniques d'obscurcissement de code}
L'analyse d'un logiciel malveillant a pour but de comprendre ses mécanismes internes : selon les logiciels il peut, entre autres, s'agir des techniques d'attaque, de communication avec le concepteur ou d'autres programmes malveillants, de clés de chiffrement utilisées. Le programmeur a donc intérêt à protéger son logiciel contre l'analyse. Son but est de la rendre plus compliquée pour nécessiter plus de ressources en temps ou en argent de la part de l'analyste.

De nombreuses techniques de protection sont applicables à un programme binaire pour rendre son analyse plus compliquée. Certaines rendent le code plus difficile à comprendre en ajoutant par exemple du code inutile. Une autre technique consiste à modifier le programme au cours de son exécution afin que le code réellement utile du binaire ne soit pas lisible à première vue : on parle alors d'auto-modification.

Un auteur de programmes malveillants peut produire dans un premier temps son programme sans protection puis utiliser un logiciel de protection ou \emph{packer} qui produit un binaire sémantiquement équivalent mais qui est rendu plus difficile à analyser. En pratique l'exécutable final, protégé, combine plusieurs techniques d'obscurcissement dont des techniques d'auto-modification.

Dans ce chapitre nous chercherons rapidement a comprendre les difficultés rencontrées pour quiconque cherche à protéger son programme contre l'analyse. Dans un second temps nous nous intéresserons aux protections rencontrées lors de l'analyse de logiciels malveillants et en particulier aux problèmes liés au chevauchement de code assembleur et à l'auto-modification.

% Nous décrivons maintenant plusieurs techniques d'obscurcissement statiques ainsi que l'auto-modification.

% $\mathcal{T}$

\section{Théorie de l'obscurcissement}
\itodo{Chaîner les protections}

Collberg et Nagra \cite{nagra2009surreptitious} définissent plusieurs propriétés souhaitables pour une protection logicielle, nous reprenons ici quelque uns de leurs arguments. La première propriété est que le programme protégé soit équivalent en terme de sorties que le programme d'origine.
Une protection, ou obscurcissement, d'un programme P est une transformation $\mathcal{T}$ telle que le programme $\mathcal{T}(P)$ a le même comportement que P : quelle que soit l'entrée I de P, $\mathcal{T}(P)(I)=P(I)$.
On souhaite également ne pas affecter de manière importante les performances du programme, que ce soit en termes de temps d'exécution ou de taille des binaires.

Afin d'évaluer l'efficacité des protections il est nécessaire de définir les actifs que l'on cherche à protéger. 
Il peut s'agir de quelques algorithmes centraux au programme, de clé de chiffrement, du nom des fonctions et des variables utilisées, etc.
Il est également utile de masquer l'utilisation de techniques de protection qui pourraient attirer l'attention lors d'une analyse antivirale. De même si l'analyste se doute que le binaire emploi des protections, il n'est pas souhaitable qu'il lui soit aisé d'identifier quelles méthodes sont employées. Ainsi dans les modifications apportées au programme il est préférable que le programme final ressemble à un programme qui aurait pu être compilé, par exemple en évitant d'utiliser des instructions exotiques rarement utilisées en pratique.


En pratique un auteur de logiciel malveillant masque les symboles utiles à l'analyse tels le nom des fonctions à la compilation. Il est utile que le vecteur de propagation réussisse à masquer qu'il est protégé, s'il l'est, pour éviter une détection prématurée par un antivirus. La charge finale et les éventuels mécanismes de communication sont eux protégés contre l'analyse.\todo{cite}

\section{Exemples d'obscurcissement}
De nombreuses techniques d'obscurcissement sont utilisées en pratique, nous en donnons quelques exemples ici.

\paragraph{Insertion de code mort.}
Insérer du non atteignable (ou code mort) peut forcer un désassembleur par parcours linéaire à se désaligner avec le code réellement exécuté et à favoriser le code mort au détriment du code légitime.
L'exemple donné en figure \ref{fig:junk_right} montre de l'assembleur avec deux octets de données placés à la suite d'un instruction \jmp. Ces deux octets aux adresses $0x08048062$ et $0x08048063$ ne sont pas atteignables. Pourtant un désassembleur linéaire (Figure \ref{fig:junk_fooled}) chercherait à les désassembler et serait alors incapable de voir une partie des instructions réellement exécutées.


\begin{figure}
\begin{lstlisting}[language={[x86masm]Assembler}, escapechar=~]
08048060    eb 02               jmp 0x8048064
08048062    0a 05		~(code non atteignable)~
08048064    83 f9 02            cmp ecx, 0x2
08048067    74 00               je 0x8048069
08048069    bb 02 00 00 00      mov ebx, 0x2 ;  0x00000002
\end{lstlisting}
\caption{Insertion de code mort dans du code légitime}
\label{fig:junk_right}
\end{figure}

\begin{figure}
\begin{lstlisting}[language={[x86masm]Assembler}, escapechar=~]
08048060    eb 02               jmp 0x8048064
08048062    0a 05 83 f9 02 74   or al, [0x7402f983]
08048068    00 bb 02 00 00 00   add [ebx+0x2], bh
\end{lstlisting}
\caption{L'insertion de code mort dupe facilement un désassemblage par parcours linéaire}
\label{fig:junk_fooled}
\end{figure}

\FloatBarrier

\paragraph{Appels de fonctions sans retour.}
L'utilisation d'un contrôle de flot non standard peut forcer un désassembleur par parcours récursif à explorer du code non atteignable. 
Le comportement par défaut de l'instruction \call\ à une adresse $a$ et de taille $n$ est d'empiler l'adresse de retour $a+n$ puis de sauter vers la première adresse de la fonction appelée.
L'instruction \ret\ placée à la fin de la fonction appelée dépile la première valeur de la pile et provoque un saut vers celle-ci.

Normalement la valeur dépilée lors du \ret\ est $a+n$ afin que le flot d'exécution revienne à la fonction appelant.
Ainsi un désassembleur récursif désassemble à partir de la cible du \call\ ainsi que de l'adresse de retour.

Une technique classique d'obscurcissement \cite{LD03}\cite{PMA} consiste à combiner l'empilement d'une adresse (\push\ $a$) et l'instruction \ret. Ces deux instructions provoquent un saut vers l'adresse $a$ sans possibilité de revenir à l'instruction suivant le \call. La transformation consiste alors à remplacer des instructions \jmp\ par la séquence \push\ puis \ret\ puisque les deux suites d'instructions suivantes sont équivalentes.
\begin{center}
\begin{tabular}{c|c}
\push\ \adr{a} & \jmp\ \adr{a}\\
\ret &
\end{tabular}
\end{center}

\paragraph{Prédicats opaques.}
% À l'instar de son comportement avec une instruction \call, 
Lorsqu'un désassembleur récursif rencontre une instruction de saut conditionnel comme \je, qui provoque un saut si les deux valeurs comparées sont égales, il cherche à désassembler à la fois la cible potentielle du saut comme l'instruction suivante, qui sera exécutée si la condition n'est pas remplie.
Une autre technique courante d'obscurcissement consiste à utiliser comme condition du saut une relation que le programmeur sait toujours vraie ou fausse \cite{MKK07}. De cette manière il prédit qu'une seule des deux branches est atteignable alors qu'un désassembleur va parcourir également la branche inutile.
Une telle condition est appelée un prédicat opaque et peut par exemple être implémentée par des relations d'arithmétique. Par exemple en appliquant le petit théorème de Fermat \cite{fermat} : quel que soit l'entier e, $e^3\ =\ e\ mod\ 3$.
Le programmeur sait que l'égalité est toujours vérifiée mais un analyseur statique ne pourra pas le déterminer aisément.

\paragraph{Applatissement de graphe de flot de contrôle.}

\section{Chevauchement de code}
\itodo{citer PMA}
\itodo{biblio + complète}
\itodo{figure telock : les octets, jolis}
On a vu que la taille d'une instruction assembleur varie de un à 15 octets \done{15 dans intel2 chercher 15}.
De plus rien n'empêche que la cible d'un saut soit une adresse se trouvant être au milieu d'une autre instruction.
Ainsi on parle de chevauchement de code lorsque deux instructions (ou plus) à des adresses différentes sont codées sur des adresses communes. Si une instruction à l'adresse \adr{a} de taille $k\geq 2$ est atteignable, il peut y avoir une autre instruction valide et atteignable à l'adresse \adr{a+1} et ces deux instructions se chevauchent.

Il est à noter que, comme indiqué par Sikorski et Honig \cite{PMA}, il n'y a dans ce cas aucun désassemblage sous forme de liste d'instructions assembleur qui soit correct puisqu'une telle liste devra choisir entre l'instruction à l'adresse \adr{a} et celle à l'adresse \adr{a+1} alors qu'elles sont toutes les deux valides et atteignables. Une solution pour écrire un tel code assembleur est de mettre la première instruction en temps qu'instruction classique tandis que la deuxième sera présente sous forme d'octets codés en dur dans le fichier assembleur.

\paragraph{Exemple dans \telock.}
Le code de la figure \ref{fig:telock_obf_disas} est extrait d'un programme protégé par \telock\ désassemblé à l'aide d'un parcours récursif à partir de l'adresse \adr{01006e7a}. Il y a une instruction \texttt{jmp +1} à l'adresse \adr{01006e7d} et codée sur les deux octets \texttt{eb ff}, qui saute vers l'adresse \adr{01006e7d+1} où est présenté l'instruction \texttt{dec ecx}, codée sur \texttt{ff c9} et qui partage donc l'octet \texttt{ff} à l'adresse \adr{01006e7d+1} avec l'instruction \jmp.

Le code assembleur permettant d'être assemblé en ces octets est donné figure \ref{fig:telock_obf_asm} : la première instruction \jmp\ peut être présente dans le code tandis que l'instruction \dec\ la chevauchant est codée en dur grâce à l'octet \texttt{c9}.

\begin{figure}
% \scriptsize
% 0x01006e73    00 0c 0b        add [ebx+ecx], cl
% 0x01006e76    80 34 0b 67     xor byte [ebx+ecx], 0x67
\begin{lstlisting}[language={[x86masm]Assembler}, escapechar=~]
01006e7a    fe 04 0b        inc byte [ebx+ecx]
01006e7d    eb ff           jmp +1
01006e7e       ff c9        dec ecx
01006e80    7f e6           jg 01006e68
01006e82    8b c1           mov eax, ecx
\end{lstlisting}
% \end{framed}
\caption{Désassemblage récursif de \telock}
\label{fig:telock_obf_disas}
\end{figure}

\begin{figure}
% \scriptsize
% 0x01006e73    00 0c 0b        add [ebx+ecx], cl
% 0x01006e76    80 34 0b 67     xor byte [ebx+ecx], 0x67
\begin{lstlisting}[language={[x86masm]Assembler}, escapechar=~]
inc byte [ebx+ecx]
jmp +1
db c9 			; ajout de l'octet c9
jg 01006e68
mov eax, ecx
\end{lstlisting}
% \end{framed}
\caption{Code assembleur du chevauchement de \telock}
\label{fig:telock_obf_asm}
\end{figure}


Le graphe de flot de contrôle correct pour ce code est donné sur la figure \ref{fig:telock_cfg}. Le sommet orange est la première instruction et les lignes en pointillés reliant deux sommets marquent un chevauchement entre les instructions de ces sommets.

\begin{figure}
\begin{center}
\includegraphics[width=0.8\textwidth]{supports/disasm/telock/telock.pdf}
\end{center}
\caption{Graphe de flot de contrôle de \telock}
\label{fig:telock_cfg}
\end{figure}

\begin{figure}
\begin{center}
\begin{tabular}{|l|c|c|c|c|c|}
\hline
Addresses & 0x01006e7d & 0x01006e7e & 0x01006e7f & 0x01006e80 & 0x01006e81\\
\hline
Bytes & eb & ff & c9 & 7f & e6\\
\hline
Layer 1 @0x01006e7d & \multicolumn{2}{c|}{jmp +1} & \cellcolor[gray]{0.0} & \multicolumn{2}{c|}{jg 0x1006e68}\\
\hline
Layer 2 @0x01006e7e & \cellcolor[gray]{0.0} & \multicolumn{2}{c|}{dec ecx} & \multicolumn{2}{c|}{\cellcolor[gray]{0.0}} \\
 \hline
% \\
\end{tabular}
\end{center}
\caption{Layers of a subset of the TELock code segment}
\label{fig:telock-layers-recursive}
\end{figure}

\begin{figure}
\begin{center}
\begin{tabular}{|l|c|c|c|c|c|c|c|c|c|c|}
\hline
Addresses & 0xf2 & 0xf3 & ... & 0xf9 & 0xfa & 0xfb & 0xfc & 0xfd & 0xfe & 0xff\\
\hline
Bytes & 79 & 07 & ... & 47 & b9 & 57 & 48 & f2 & ae & 55\\
\hline
Layer 1 @0xf2 & \multicolumn{2}{c|}{jns +9 (0xfb)} & ... & inc edi & \multicolumn{5}{c|}{mov ecx, aef24857} & push ebp\\
\hline
Layer 2 @0xfb & \multicolumn{5}{c|}{\cellcolor[gray]{0.0}} & push edi & dec eax & \multicolumn{2}{c|}{repne scasb} & \cellcolor[gray]{0.0}\\
\hline
% \\
\end{tabular}
\end{center}
\caption{Layers of a subset of the UPX code segment}
\label{fig:upx-layers-recursive}
\end{figure}

\paragraph{Exemple dans UPX.}

\section{Auto-modification}

% \begin{figure}
% \begin{center}
% \begin{tabular}{|l|l|l|}
% \hline
% Adresse & Octets & Instruction\\
% \hline
%  8048060  &  (...)         	& Pile -> RWX \\ 
%  804807c  &  bf 00 00 00 00         &  mov    edi, 0x0 \\
%  8048081  &  b8 91 80 04 08         &  mov    eax, 0x8048091 \\
%  8048086  &  66 c7 00 eb 00         &  mov    [eax], 0xeb \\
%  804808b  &  66 c7 40 01 07 00      &  mov    [eax+1], 0x7 \\
%  8048091  &  eb 0e                  &  jmp    80480a1 <edi3> \\
%  8048093  &  bf 01 00 00 00         &  mov    edi,0x1 \\
%  8048098  &  eb 0e                  &  jmp    80480a8 <fin> \\
%  804809a  &  bf 02 00 00 00         &  mov    edi,0x2 \\
%  804809f  &  eb 07                  &  jmp    80480a8 <fin> \\
%  80480a1  &  bf 03 00 00 00         &  mov    edi,0x3 \\
%  80480a6  &  eb 00                  &  jmp    80480a8 <fin> \\
%  80480a8  &  (...)		    &  Affiche edi \\
%  80480c3  &  (...)		    & Quitte \\
% \hline
% \end{tabular}
% \end{center}
% \caption{Exemple de code auto-modifiant}
% \label{fig:unevague_v0_code}
% \end{figure}

\begin{figure}
\begin{center}
\subfigure[Code assembleur]{
\begin{tabular}[b]{|l|l|l|}
\hline
Adresse & Octets & Instruction\\ 
\hline
 8048060  &  (...)         	& Pile -> RWX \\ 
 804807c  &  bf 00 00 00 00         &  mov    edi, 0x0 \\
 8048081  &  b8 91 80 04 08         &  mov    eax, 0x8048091 \\
 8048086  &  66 c7 00 eb 00         &  mov    [eax], 0xeb \\
 804808b  &  66 c7 40 01 07 00      &  mov    [eax+1], 0x7 \\
 8048091  &  eb 0e                  &  jmp    80480a1 <edi3> \\
 8048093  &  bf 01 00 00 00         &  mov    edi,0x1 \\
 8048098  &  eb 0e                  &  jmp    80480a8 <fin> \\
 804809a  &  bf 02 00 00 00         &  mov    edi,0x2 \\
 804809f  &  eb 07                  &  jmp    80480a8 <fin> \\
 80480a1  &  bf 03 00 00 00         &  mov    edi,0x3 \\
 80480a6  &  eb 00                  &  jmp    80480a8 <fin> \\
 80480a8  &  (...)		    &  Affiche edi \\
 80480c3  &  (...)		    & Quitte \\
\hline
\end{tabular}
\label{fig:unevague_v0_code}
}
\subfigure[Graphe de flot de contrôle]{
\includegraphics[width=1.0\textwidth]{supports/unevague/uv.pdf}
\label{fig:unevague_v0_cfg}
}
\end{center}
\ijym{détailler les ... (en annexe?)}
\ijym{fonction f, adresses f, f+1, f+3, ...}
\caption{Code assembleur auto-modifiant}
\label{fig:unevague_v0}
\end{figure}

Il a été expliqué dans la section \ref{section:assembleur} que, avec l'architecture de Harvard modifiée, le code n'est pas physiquement séparé des données lors de l'exécution sur une machine réelle.
Un programme auto-modifiant est simplement un programme utilisant cette propriété pour modifier le code assembleur le définissant au cours même de son exécution.
Ainsi on parle de comportement auto-modifiant lorsqu'une instruction du programme est codée sur au moins un octet qui a au préalable été modifié par ce programme.

En pratique les processeurs récents implémentent une protection, appelée bit NX ou W\textasciicircum X (prononcé ``W xor X''), permettant d'empêcher qu'une page mémoire puisse être à la fois écrite et exécutée lors de l'exécution du programme.
Cette protection a été ajoutée pour éviter des attaques résultant en l'exécution de code dans des données écrites par l'utilisateur du programme et non pour interdire l'auto-modification qui a des cas d'utilisation légitimes.
De ce fait l'activation ou non de la protection est spécifiée lors de la compilation et si un programme n'est pas protégé il lui suffit d'utiliser un appel système (\texttt{mprotect} sous linux) pour autoriser l'exécution de code sur la pile.

Prenons le programme de la figure \ref{fig:unevague_v0_code}. Ce programme commence par autoriser l'accès en écriture à la la section de code \ptext\ puis écrit sur la pile, modifie la valeur du registre \edi\ et termine par l'affichage de la valeur de \edi.
Si on ne prend pas en compte l'écriture sur la pile, il semble évident au vu du graphe de flot de contrôle (Figure \ref{fig:unevague_v0_cfg}), vu que la première instruction de saut provoque un saut vers l'instruction \texttt{mov edi,0x3} et que la seconde provoque l'affichage de \edi, que la valeur finale du registre est 3.
Pourtant les instructions \texttt{mov [eax],0xeb} et \texttt{mov [eax+1],0x07} aux l'adresse $0x8048086$ et $0x804808b$ remplacent le saut initial par un saut vers l'adresse $0x8048098$ où la valeur de \edi\ sera fixée à 2 avant l'affichage de celle-ci.


On constate ici que le programme se modifie au cours de son exécution et donc
\begin{itemize}
 \item On ne peut pas se contenter de la représentation d'origine du programme pour l'analyser.
 \item Le graphe de flot de contrôle initial peut être amené à évoluer au cours de l'exécution du programme.
\end{itemize}

\section{Logiciels malveillants et obscurcissement}
\itodo{packers, comment chaîner les obfuscations}
\itodo{utiliser des petites variations d'un packer pour éviter la détection, parler de la détection}

\section{Conclusion}
Cette thèse s'intéresse particulièrement à l'analyse des programmes écrits en assembleur \xq\ et \xs. Ces programmes ont en général été compilés à partir d'un langage de haut niveau puis ont été modifiés à l'aide d'un logiciel de protection. Les binaires que l'on étudie sont donc protégés avec des techniques statiques comme auto-modifiantes. Notre travail consiste alors à chercher à désassembler correctement ces programmes dans le but de faciliter leur analyse.

Les chapitres suivants détailleront plusieurs techniques d'analyse que nous avons appliquées. Nous nous intéresserons d'abord à l'aspect auto-modifiant des programmes et verrons comment l'analyse dynamique peut être utilisée pour reconstruire un modèle pour le programme auto-modifiant. Dans un second temps nous introduiront des techniques d'analyse statiques pour chercher à recomposer le maximum de code assembleur du programme et contourner d'autres méthodes de protection comme le chevauchement de code.

\DontFrameThisInToc
\chapter{Sémantique de l'assembleur et langage intermédiaire}
\input{4-semantique}

\DontFrameThisInToc
\chapter{Désassemblage}
\input{5-desassembleur}



% \DontFrameThisInToc
% \chapter{Analyse dynamique}
% \input{6-analyse-dynamique}

\DontFrameThisInToc
\chapter{Analyse statique et hybride}
\input{6-analyse-statique}


% \DontFrameThisInToc
% \chapter{Résultats}
% Ici on parle d'assembleur

\section{Premiere section}
Ahah

% \WriteThisInToc
% \FrameThisInToc
% \NumberThisInToc
\part{Analyse morphologique}
\DontFrameThisInToc
\chapter{Analyse de binaires et comparaison de graphes}
\section{État de l'art: Technique de détection}
\section{Faiblesse des approches syntaxiques}
\section{Comparaison des GFC}

\DontFrameThisInToc
\chapter{Algorithmes de détection de sous-graphes et approximations}
\todo[inline]{Perfs + FP/FN}
\todo[inline]{Limitations : obfuscation du CFG}

% \DontFrameThisInToc
% \chapter{Validation de l'approche}


\DontFrameThisInToc
\chapter{Application à la détection de librairies}
\todo[inline]{Cas Waledac / OpenSSL}

\DontFrameThisInToc
\chapter{Cas d'étude : Duqu et Stuxnet}

\PutLineInToc
% \PutNewPageInToc

%-------------------------------------------------------------------
%              L'index (toujours sur deux colonnes)
%-------------------------------------------------------------------
\BeginIndWith{Voici un index}
\PrintIndex

\onecolumn

%-------------------------------------------------------------------
%                       La bibliographie
%-------------------------------------------------------------------

% La bibliographie (comme d'habitude)

\nocite{*}
% \bibliographystyle{unsrt}
% \bibliographystyle{named}
% \bibliography{these}
\printbibliography

\NumberAbstractPages
\begin{ThesisAbstract}
  \begin{FrenchAbstract}
    Résumé.
    \KeyWords{a, b, c.}
  \end{FrenchAbstract}
  \begin{EnglishAbstract}
    Abstract.
    \KeyWords{a, b, c.}
  \end{EnglishAbstract}
\end{ThesisAbstract}


\end{document}


