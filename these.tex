%             Exemple d'utilisation de la classe thesul
%             ------------------------------------------
%
%
% (de manière generale, les commandes de thesul sont celles
% qui ne sont pas complètement en minuscules)
%
% Voir la documentation complète pour plus de détails.
%
%
% D. Roegel, 30 mars 2013
%
\documentclass[10pt, oneside]{TUL/thesul}
% \documentclass[12pt]{TUL/thesul}
%----------------------------------------------------------------------
%                     Chargement de quelques packages
%----------------------------------------------------------------------

% Si l'on veut produire une version PDF avec des hyperliens :
% \usepackage[pageanchor=false]{TUL/tulhypref}
\usepackage[hidelinks, pdftex]{TUL/tulhypref}

% Si on veut le style de bibliographie named :
%\usepackage{named}

% Pour les figures :
\usepackage{graphicx}

% Si on veut des mini-tables des matières (utiliser minitoc-hyper 
% en conjonction avec tulhypref) :
\usepackage[french]{minitoc}

\usepackage[backgroundcolor=blue!20!white, linecolor=black]{todonotes}
% \usepackage[disable]{todonotes}
\usepackage{titlesec}
\usepackage{url}
\usepackage{listings}
\usepackage{pstricks}
\usepackage{subfigure}
\usepackage[backend=biber, language=french, maxnames=5, citestyle=alphabetic,bibstyle=alphabetic,backref,abbreviate=false,dateabbrev=false,isbn=false,url=false,doi=true]{biblatex}

\addbibresource{these.bib}

% Structure
\newtheorem{pb}{Problème}
\newtheorem{theo}{Théorème}
\newtheorem{defi}{Définition}
\newtheorem{prop}{Proposition}
% \theoremstyle{remark}\newtheorem*{preuve}{Preuve}

% % Abbreviations :
\newcommand{\helloworld}{\texttt{Hello World}}
\newcommand{\nasm}{NASM}
\newcommand{\xq}{x86}
\newcommand{\xs}{x86$\_$64}

% Registres
\newcommand{\eax}{\texttt{eax}}
\newcommand{\ebx}{\texttt{ebx}}
\newcommand{\ecx}{\texttt{ecx}}

% Instructions
\newcommand{\mov}{\texttt{mov}}
\newcommand{\cmp}{\texttt{cmp}}
\newcommand{\add}{\texttt{add}}

\titleformat{\chapter}[display]
  {\bfseries\huge}
  {\filleft\Large\chaptertitlename~\thechapter}
  {1ex}
  {\titlerule\vspace{1.5ex}\filright}
  [\vspace{1ex}\titlerule]

%-------------------------------------------------------------------
%                             Marges
%-------------------------------------------------------------------

% pour positionner les vraies marges:
%\SetRealMargins{1mm}{1mm}

%-------------------------------------------------------------------
%                             En-têtes
%-------------------------------------------------------------------

% Les en-têtes: quelques exemples
%\UppercaseHeadings 
%\UnderlineHeadings
%\newcommand\bfheadings[1]{{\bf #1}}
%\FormatHeadingsWith{\bfheadings}
%\FormatHeadingsWith{\uppercase}
%\FormatHeadingsWith{\underline}
\newcommand\upun[1]{\uppercase{\underline{\underline{#1}}}}
\FormatHeadingsWith\upun

\newcommand\itheadings[1]{\textit{#1}}
\FormatHeadingsWith{\itheadings}

% pour avoir un trait sous l'en-tete:
\setlength{\HeadRuleWidth}{0.4pt}

%-------------------------------------------------------------------
%                         Les références
%-------------------------------------------------------------------

\NoChapterNumberInRef
\NoChapterPrefix

%-------------------------------------------------------------------
%                           Brouillons
%-------------------------------------------------------------------

% ceci ajoute une marque « brouillon » et la date
%\ThesisDraft

%-------------------------------------------------------------------
%                   Pour collecter un glossaire et un index
%-------------------------------------------------------------------

\makeglossary
\makeindex

\begin{document}





\OddHead={{\leftmark\rightmark}{\hfil\slshape\rightmark}}
\EvenHead={{\leftmark}{{\slshape\leftmark}\hfil}}
\OddFoot={\hfil\thepage}
\EvenFoot={\thepage\hfil}
\pagestyle{ThesisHeadingsII}

%-------------------------------------------------------------------
%                          Encadrements
%-------------------------------------------------------------------

% encadre les chapitres dans la table des matières:
% (ces commandes doivent figurer apres \begin{document}

\FrameChaptersInToc  
%\FramePartsInToc


%-------------------------------------------------------------------
%            Réinitialisation de la numérotation des chapitres
%-------------------------------------------------------------------

% Si la commande suivante est présente,
% elle doit figurer APRÈS \begin{document}
% et avant la première commande \part
% \ResetChaptersAtParts 

%-------------------------------------------------------------------
%               mini-tables des matières par chapitre
%-------------------------------------------------------------------

% préparer les mini-tables des matières par chapitre.
% (commande de minitoc.sty)
\dominitoc

%-------------------------------------------------------------------
%                         Page de titre:
%-------------------------------------------------------------------

\ThesisTitle{Analyse morphologique de logiciels malveillants auto-modifiants}
\ThesisDate{dd/mm/aaaa}
\ThesisAuthor{Aurélien Thierry}

% Type de la these
\ThesisUL

% Jury:

% (ne pas mettre de \\ apres la dernière entree)

% Exemple de création d'une nouvelle catégorie dans le jury:

\NewJuryCategory{family}{\it Membre de la famille :}
                        {\it Membres de la famille :}

\family={Mon frère\\Ma sœur}

\def\blanc{\hspace*{1cm}}

\President    = {Le président}
\Rapporteurs  = {Le rapporteur 1&de Paris\\
                 Le rapporteur 2\\
                 \blanc suite&taratata\\
                 Le rapporteur 3}
\Examinateurs = {L'examinateur 1&d'ici\\
                 L'examinateur 2}
%\Invites=       {}

% Création de la page de titre:
\MakeThesisTitlePage

% on peut en faire plusieurs:
%\MakeThesisTitlePage

%-------------------------------------------------------------------


%-------------------------------------------------------------------
%                          remerciements
%-------------------------------------------------------------------

%\DontFrameThisInToc
\begin{ThesisAcknowledgments}
Merci.
\end{ThesisAcknowledgments}

%-------------------------------------------------------------------
%                            dédicace
%-------------------------------------------------------------------

% \begin{ThesisDedication}
% Ah.
% \end{ThesisDedication}


%-------------------------------------------------------------------
%                  écriture de `Chapitre' et `Partie' 
%                      dans la table des matières
%-------------------------------------------------------------------

\WritePartLabelInToc
\WriteChapterLabelInToc

%-------------------------------------------------------------------
%                        table des matières
%-------------------------------------------------------------------

\tableofcontents

%-------------------------------------------------------------------
%              Exemple d'utilisation de \SpecialSection
%-------------------------------------------------------------------

% \SpecialSection{Introduction générale}


% Pour ne pas avoir le mot « Chapitre » au début de chaque chapitre.
% \NoChapterHead

\DontWriteThisInToc   
\listoffigures


% La commande \mainmatter permet de passer
% à la numérotation arabe (ce que fait \pagenumbering{arabic}) 
% et de faire commencer la nouvelle page 1 sur une page impaire.
% On évitera donc d'utiliser directement \pagenumbering{arabic}.
\mainmatter

\DontFrameThisInToc
\NumberThisInToc
\chapter{Introduction}
% Ceci est une introduction.\todo{Mettre une intro}
% \todo[inline]{Une bonne introduction.}
\todo[inline]{Contexte, enjeux\\ Buts de la thèse\\ Contributions}

% \section{Contexte et enjeux}
\todo[inline]{Contexte, enjeux:}
\todo[inline]{Malware, sécurité, détection}
% Un logiciel opérant sur un ordinateur est dit malveillant s'il réalise volontairement des opérations allant à l'encontre de l'intérêt de l'utilisateur et ce à l'insu de celui-ci. 
Un logiciel malveillant réalise volontairement des opérations allant à l'encontre de l'intérêt de l'utilisateur. 
Certaines de ces actions peuvent être objectivement malveillantes comme c'est le cas avec Stuxnet qui vise des éléments d'un système de contrôle industriel afin de le rendre inutilisable. 
% Les administrateurs utilisent quotidiennement des programmes de gestion de machine à distance. 
Un logiciel malveillant visant à fournir à un attaquant un accès à distance ne diffère pas fondamentalement, en termes de fonctionnalités, d'un logiciel légitime utilisé par un administrateur.
La différence entre les deux est que l'administrateur n'est pas au courant de l'installation du logiciel malveillant et ne peut pas le contrôler.
Nous proposons donc la définition suivante pour un logiciel malveillant.
\begin{defi}
Un logiciel est dit malveillant s'il réalise volontairement des opérations allant à l'encontre de l'intérêt de l'utilisateur, et ce à l'insu de celui-ci.
\end{defi}

Les actions malveillantes peuvent être de trois nature. On parle d'atteinte à la confidentialité lorsque des données privées sont obtenues par l'attaquant, d'atteinte à l'intégrité lorsque de l'information présente sur le système attaqué est modifiée par l'attaquant et d'atteinte à la disponibilité du système si l'attaquant rend le système inutilisable ou le surcharge\todo{dispo?}.

Les logiciels malveillants s'attaquent en général à la confidentialité et la disponibilité du système bien que les outils d'administration à distance soient capables, une fois la machine infectée, d'attaquer les trois aspects de la sécurité de la machine.

Ainsi, en général, un logiciel malveillant diffère d'un logiciel légitime en ce qu'il cherche à dissimuler son existence et son action sur le système. Il déploie à cette fin des techniques de protection logicielles rendant son analyse plus difficile que celle du programme légitime.


\section{Détection de logiciels malveillants}
\todo[inline]{+ anti-détection}
Les premiers travaux traitant de virologie informatique datent de 1986 avec Cohen \cite{Cohen86} puis Adleman \cite{Adleman88} en 1988. Ils s'intéressent particulièrement au comportement auto-reproducteur de certains programmes. Adleman \todo{vrai?} aboutit au résultat suivant dans le cas des logiciels malveillants : la classification d'un logiciel comme malveillant est un problème $\Pi_2$ complet, soit plus difficile que le problème de l'arrêt qui est déjà indécidable.

Les décennies qui ont suivi ont vu apparaître différentes techniques de détection partielles dont la plus répandue est l'analyse de signatures syntaxiques. Les approches par signatures consistent dans un premier temps à extraire d'un corpus de logiciels malveillants connus des caractéristiques comme certaines instructions ou plus généralement une expression rationnelle. Dans un second temps, pour classifier un logiciel inconnu, on regarde s'il possèdent une des caractéristiques extraites du corpus.
Cette technique, dans le cas des expressions rationnelles, possède le double avantage d'être rapide et de générer peu de fausses alarmes.

Cependant chaque souche originale d'un logiciel malveillant est généralement déclinée en de nombreuses versions dont les fonctionnalités varient. Ces souches formes une famille de logiciels malveillants. Pour éviter la détection par signatures syntaxiques, il suffit souvent d'insérer du code inutile ou de réorganiser le code.

Nous nous sommes alors intéressés à la technique de détection initiée par Kaczmarek \cite{AThierry_BKM08} lors de sa thèse. Il s'agit d'une technique de détection par signatures basée sur la comparaison de graphes de flots de contrôles, c'est à dire du graphe structurant l'exécution du logiciel.
\todo[inline]{exemple prog -> CFG}

Dans la pratique, les logiciels malveillants sont souvent protégés par une technique d'encapsulation cachant complètement la charge utile à un analyste 

\section{Problèmes de recherche}


\section{Organisation du document}

% \WriteThisInToc
% \FrameThisInToc
% \NumberThisInToc
\part{Désassemblage et analyse de binaires}

% \NumberThisInToc
\DontFrameThisInToc
\chapter{Assembleur}
Nous nous intéressons en premier lieu aux programmes malveillants fonctionnant sur des ordinateurs personnels.
Les programmes s'exécutant sur ces machines sont compilés afin d'être exécutés nativement dans le langage assembleur spécifique au processeur de la machine.


\section{Malwares binaires}

\section{Assembleur x86 et x86$\_$64}
L'architecture la plus fréquente sur ces ordinateurs personnels est celle des processeurs Intel CISC

\section{Obfuscations}


\DontFrameThisInToc
\chapter{Désassemblage et analyse statique}
\todo[inline]{Code + CFG\\ Sémantique, Layers\\ sémantique SM: SM-VSA + emulation BAP}
% Ici on parle d'assembleur

\section{Premiere section}
Ahah

\DontFrameThisInToc
\chapter{Analyse dynamique et hybride}
\todo[inline]{Limites analyse statique, analyse dynamique et waves\\ disasm.py}
% Ici on parle d'assembleur

\section{Premiere section}
Ahah

\DontFrameThisInToc
\chapter{Résultats}
% Ici on parle d'assembleur

\section{Premiere section}
Ahah

% \WriteThisInToc
% \FrameThisInToc
% \NumberThisInToc
\part{Analyse morphologique}
\DontFrameThisInToc
\chapter{Analyse de graphes et algorithmes}

\DontFrameThisInToc
\chapter{Validation de l'approche}
\todo[inline]{Perfs + FP/FN}

\DontFrameThisInToc
\chapter{Application à la détection de librairies}
\todo[inline]{Cas Waledac / OpenSSL}

\DontFrameThisInToc
\chapter{Cas d'étude : Duqu et Stuxnet}

\PutLineInToc
% \PutNewPageInToc

%-------------------------------------------------------------------
%              L'index (toujours sur deux colonnes)
%-------------------------------------------------------------------
\BeginIndWith{Voici un index}
\PrintIndex

\onecolumn

%-------------------------------------------------------------------
%                       La bibliographie
%-------------------------------------------------------------------

% La bibliographie (comme d'habitude)

\nocite{*}
% \bibliographystyle{unsrt}
% \bibliographystyle{named}
% \bibliography{these}
\printbibliography

\NumberAbstractPages
\begin{ThesisAbstract}
  \begin{FrenchAbstract}
    Résumé.
    \KeyWords{a, b, c.}
  \end{FrenchAbstract}
  \begin{EnglishAbstract}
    Abstract.
    \KeyWords{a, b, c.}
  \end{EnglishAbstract}
\end{ThesisAbstract}


\end{document}


