%             Exemple d'utilisation de la classe thesul
%             ------------------------------------------
%
%
% (de manière generale, les commandes de thesul sont celles
% qui ne sont pas complètement en minuscules)
%
% Voir la documentation complète pour plus de détails.
%
%
% D. Roegel, 30 mars 2013
%
\documentclass[11pt]{TUL/thesul}
%----------------------------------------------------------------------
%                     Chargement de quelques packages
%----------------------------------------------------------------------

% Si l'on veut produire une version PDF avec des hyperliens :
%\usepackage[pageanchor=false]{tulhypref}

% Si on veut le style de bibliographie named :
%\usepackage{named}

% Pour les figures :
\usepackage{graphicx}

% Si on veut des mini-tables des matières (utiliser minitoc-hyper 
% en conjonction avec tulhypref) :
\usepackage[french]{minitoc}


%-------------------------------------------------------------------
%                             Marges
%-------------------------------------------------------------------

% pour positionner les vraies marges:
%\SetRealMargins{1mm}{1mm}

%-------------------------------------------------------------------
%                             En-têtes
%-------------------------------------------------------------------

% Les en-têtes: quelques exemples
%\UppercaseHeadings 
%\UnderlineHeadings
%\newcommand\bfheadings[1]{{\bf #1}}
%\FormatHeadingsWith{\bfheadings}
%\FormatHeadingsWith{\uppercase}
%\FormatHeadingsWith{\underline}
\newcommand\upun[1]{\uppercase{\underline{\underline{#1}}}}
\FormatHeadingsWith\upun

\newcommand\itheadings[1]{\textit{#1}}
\FormatHeadingsWith{\itheadings}

% pour avoir un trait sous l'en-tete:
\setlength{\HeadRuleWidth}{0.4pt}

%-------------------------------------------------------------------
%                         Les références
%-------------------------------------------------------------------

\NoChapterNumberInRef
\NoChapterPrefix

%-------------------------------------------------------------------
%                           Brouillons
%-------------------------------------------------------------------

% ceci ajoute une marque « brouillon » et la date
%\ThesisDraft

%-------------------------------------------------------------------
%                   Pour collecter un glossaire et un index
%-------------------------------------------------------------------

\makeglossary
\makeindex

\begin{document}


      \OddHead={{\leftmark\rightmark}{\hfil\slshape\rightmark}}
      \EvenHead={{\leftmark}{{\slshape\leftmark}\hfil}}
      \OddFoot={\hfil\thepage}
      \EvenFoot={\thepage\hfil}
      \pagestyle{ThesisHeadingsII}

%-------------------------------------------------------------------
%                          Encadrements
%-------------------------------------------------------------------

% encadre les chapitres dans la table des matières:
% (ces commandes doivent figurer apres \begin{document}

\FrameChaptersInToc  
%\FramePartsInToc


%-------------------------------------------------------------------
%            Réinitialisation de la numérotation des chapitres
%-------------------------------------------------------------------

% Si la commande suivante est présente,
% elle doit figurer APRÈS \begin{document}
% et avant la première commande \part
\ResetChaptersAtParts 

%-------------------------------------------------------------------
%               mini-tables des matières par chapitre
%-------------------------------------------------------------------

% préparer les mini-tables des matières par chapitre.
% (commande de minitoc.sty)
\dominitoc

%-------------------------------------------------------------------
%                         Page de titre:
%-------------------------------------------------------------------

\ThesisTitle{Comment j'ai r\'eussi \`a prouver mon existence}
\ThesisDate{28 janvier 1986}
\ThesisAuthor{Auteur anonyme}

% Type de la these
\ThesisUL

% Jury:

% (ne pas mettre de \\ apres la dernière entree)

% Exemple de création d'une nouvelle catégorie dans le jury:

\NewJuryCategory{family}{\it Membre de la famille :}
                        {\it Membres de la famille :}

\family={Mon frère\\Ma sœur}

\def\blanc{\hspace*{1cm}}

\President    = {Le président}
\Rapporteurs  = {Le rapporteur 1&de Paris\\
                 Le rapporteur 2\\
                 \blanc suite&taratata\\
                 Le rapporteur 3}
\Examinateurs = {L'examinateur 1&d'ici\\
                 L'examinateur 2}
%\Invites=       {}

% Création de la page de titre:
\MakeThesisTitlePage

% on peut en faire plusieurs:
%\MakeThesisTitlePage

%-------------------------------------------------------------------


%-------------------------------------------------------------------
%                          remerciements
%-------------------------------------------------------------------

%\DontFrameThisInToc
\begin{ThesisAcknowledgments}
Merci.
\end{ThesisAcknowledgments}

%-------------------------------------------------------------------
%                            dédicace
%-------------------------------------------------------------------

\begin{ThesisDedication}
Je dédie cette thèse\\
à ma machine.\\
Oui, à Pandore,\\
qui fut la première de toutes.
\end{ThesisDedication}


%-------------------------------------------------------------------
%                  écriture de `Chapitre' et `Partie' 
%                      dans la table des matières
%-------------------------------------------------------------------

\WritePartLabelInToc
\WriteChapterLabelInToc

%-------------------------------------------------------------------
%                        table des matières
%-------------------------------------------------------------------

\tableofcontents

%-------------------------------------------------------------------
%              Exemple d'utilisation de \SpecialSection
%-------------------------------------------------------------------

% \SpecialSection{Introduction générale}


% Pour ne pas avoir le mot « Chapitre » au début de chaque chapitre.
\NoChapterHead

\DontWriteThisInToc   
\listoffigures


% La commande \mainmatter permet de passer
% à la numérotation arabe (ce que fait \pagenumbering{arabic}) 
% et de faire commencer la nouvelle page 1 sur une page impaire.
% On évitera donc d'utiliser directement \pagenumbering{arabic}.
\mainmatter

\DontFrameThisInToc
\NumberThisInToc
\chapter{Introduction}

\WriteThisInToc
\FrameThisInToc
\NumberThisInToc
\part{Désassemblage et analyse de binaires}

% \NumberThisInToc
\DontFrameThisInToc
\chapter{Assembleur}
Ici on parle d'assembleur

\section{Premiere section}
Ahah

\WriteThisInToc
\FrameThisInToc
\NumberThisInToc
\part{Analyse morphologique}
\DontFrameThisInToc
\chapter*{Second chapitre}
\section{Premiere section}

Une autre page avec plein de texte très varié.
Une autre page avec plein de texte très varié.

\PutLineInToc
% \PutNewPageInToc

%-------------------------------------------------------------------
%              L'index (toujours sur deux colonnes)
%-------------------------------------------------------------------
\BeginIndWith{Voici un index}
\PrintIndex

\onecolumn

%-------------------------------------------------------------------
%                       La bibliographie
%-------------------------------------------------------------------

% La bibliographie (comme d'habitude)

\nocite{*}
\bibliographystyle{named}
\bibliography{these}


\NumberAbstractPages
\begin{ThesisAbstract}
  \begin{FrenchAbstract}
    Résumé.
    \KeyWords{a, b, c.}
  \end{FrenchAbstract}
  \begin{EnglishAbstract}
    Abstract.
    \KeyWords{a, b, c.}
  \end{EnglishAbstract}
\end{ThesisAbstract}


\end{document}


