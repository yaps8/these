%             Exemple d'utilisation de la classe thesul
%             ------------------------------------------
%
%
% (de manière generale, les commandes de thesul sont celles
% qui ne sont pas complètement en minuscules)
%
% Voir la documentation complète pour plus de détails.
%   
%
% D. Roegel, 30 mars 2013
%
% \documentclass[12pt, oneside]{TUL/thesul}
\documentclass[12pt]{TUL/thesul}
%----------------------------------------------------------------------
%                     Chargement de quelques packages
%----------------------------------------------------------------------

% Si l'on veut produire une version PDF avec des hyperliens :
% \usepackage[pageanchor=false]{TUL/tulhypref}
% \usepackage[hidelinks, pdftex]{TUL/tulhypref}
\usepackage{etex}
\usepackage[hidelinks]{TUL/tulhypref}
% \usepackage[sc]{mathpazo}
\linespread{1.0}
% Si on veut le style de bibliographie named :
%\usepackage{named}

% Pour les figures :
\usepackage{graphicx}

% Si on veut des mini-tables des matières (utiliser minitoc-hyper 
% en conjonction avec tulhypref) :
\usepackage[french]{minitoc}

\usepackage{titlesec}
\usepackage{url}
\usepackage{listings}
\usepackage{pstricks}
\usepackage{subfigure}
\usepackage{amsmath}
\usepackage{amsthm}
\usepackage{amssymb}
\usepackage{tabularx}
\usepackage{textcomp}
\usepackage{multirow}
\usepackage[algoruled,french,onelanguage,algochapter]{algorithm2e}
\usepackage[section]{placeins}
\usepackage{xcolor}
\usepackage{colortbl}
\usepackage{longtable}
\usepackage{booktabs}
\setlength{\aboverulesep}{0pt}
\setlength{\belowrulesep}{0pt}

\usepackage{epstopdf}
\usepackage{graphicx} % pour insérer des images
\usepackage{stmaryrd}
\usepackage{amsfonts}
% \usepackage{tikz}
% \usepackage{tikz-qtree}
% les figure imbriquées
\usepackage{epsfig}
% \usepackage{enumerate}
\usepackage{pifont}

\usepackage{pgf}
\usepackage{tikz, calc}
% \usepackage{tikz-cd}
\usetikzlibrary{positioning}
\usetikzlibrary{fit}
\usetikzlibrary{shapes.multipart,calc}
\usetikzlibrary{arrows}

% \usepackage[math]{iwona} 
% \usepackage{iwona} 
% \SetMathAlphabet{\mathtt}{iwona}{OT1}{\ttdefault}{m}{n}

\usepackage[backend=biber, language=french, maxnames=10, citestyle=alphabetic,bibstyle=alphabetic,backref,abbreviate=false,dateabbrev=false,isbn=false,url=false,doi=true]{biblatex}
% \usepackage[backend=biber, language=french, maxnames=5,backref,abbreviate=false,dateabbrev=false,isbn=false,url=false,doi=true]{biblatex}
\addbibresource{these.bib}
% \usepackage[font=small,skip=0pt]{caption}
\usepackage[skip=0pt]{caption}
\usepackage{etoolbox}
\usepackage{needspace}

% Todo
\newcommand{\done}[1]{\todo[color=green!80!blue!80]{#1}}
\newcommand{\idone}[1]{\todo[inline,color=green!80!blue!80]{#1}}

% Samples
\newcommand{\telock}{tELock}

% Structure
\newtheorem{pb}{Problème}
\newtheorem{theo}{Théorème}
\newtheorem{defi}{Définition}
\newtheorem{prop}{Proposition}
\newtheorem{propri}{Propriété}
\newtheorem{pr}{Preuve}
\newtheorem{cor}{Corollaire}
\newtheorem{rem}{Remarque}
% \theoremstyle{remark}\newtheorem*{preuve}{Preuve}

% % Abbreviations :
\newcommand{\helloworld}{\texttt{Hello World}}
\newcommand{\nasm}{NASM}
\newcommand{\xq}{x86}
\newcommand{\xs}{x86$\_$64}

% Adresses mémoire :
\newcommand{\adr}[1]{$#1$}

% Registres
\newcommand{\eax}{\texttt{eax}}
\newcommand{\ebx}{\texttt{ebx}}
\newcommand{\ecx}{\texttt{ecx}}
\newcommand{\edi}{\texttt{edi}}

% Instructions
\newcommand{\mov}{\texttt{mov}}
\newcommand{\cmp}{\texttt{cmp}}
\newcommand{\add}{\texttt{add}}
\newcommand{\jmp}{\texttt{jmp}}
\newcommand{\ret}{\texttt{ret}}
\newcommand{\call}{\texttt{call}}
\newcommand{\push}{\texttt{push}}
\newcommand{\jne}{\texttt{jne}}
\newcommand{\je}{\texttt{je}}
\newcommand{\sub}{\texttt{sub}}

% Sémantique statique
\newcommand{\BN}{\mathbb N}
\newcommand{\BV}{\mathbb V}
\newcommand{\BX}{\mathbb X}
\newcommand{\BA}{\mathbb A}
\newcommand{\BP}{\mathbb P}
\newcommand{\BT}{\mathbb T}
\newcommand{\BL}{\mathbb L}
\newcommand{\BB}{\mathbb B}
\newcommand{\BNB}{\mathbb N\cup\{\bot\}}
\newcommand{\PN}{\mathcal{P}(\BN)}
\newcommand{\PMN}{\mathcal{P}_M(\BN)}
\newcommand{\Trs}{\mathbb N\cup\{\bot,\top\}}
\newcommand{\TTrs}{\mathcal P(\mathbb V\rightarrow\mathbb N\cup\{\bot,\top\})}
\newcommand{\Tr}{\PN\cup{\top,\bot}}
\newcommand{\TrM}{\PMN\cup\{\top,\bot\}}
\newcommand{\si}{\sigma_{init}}
\newcommand{\specialcell}[2][c]{%
  \begin{tabular}[#1]{@{}l@{}}#2\end{tabular}}
  
% Sémantique dynamique
\newcommand{\CA}{\mathcal A}
\newcommand{\CI}{\mathcal I}
\newcommand{\CC}{\mathcal C}
\newcommand{\CR}{\mathcal R}
\newcommand{\CW}{\mathcal W}
\newcommand{\da}[1]{$\CA[#1]$}
\newcommand{\di}[1]{$\CI[#1]$}
\newcommand{\dc}[1]{$\CC[#1]$}
\newcommand{\dr}[1]{$\CR[#1]$}
\newcommand{\dw}[1]{$\CW[#1]$}


\DefineBibliographyStrings{french}{
        in = {},%
%       in = {\emph{Dans}}%
        backrefpage = {cité page},
        backrefpages = {cité pages}
}

\titleformat{\chapter}[display]
  {\bfseries\huge}
  {\filleft\Large\chaptertitlename~\thechapter}
  {1ex}
  {\titlerule\vspace{1.5ex}\filright}
  [\vspace{1ex}\titlerule]

%-------------------------------------------------------------------
%                             Marges
%-------------------------------------------------------------------

% pour positionner les vraies marges:
%\SetRealMargins{1mm}{1mm}

%-------------------------------------------------------------------
%                             En-têtes
%-------------------------------------------------------------------

% Les en-têtes: quelques exemples
%\UppercaseHeadings 
%\UnderlineHeadings
%\newcommand\bfheadings[1]{{\bf #1}}
%\FormatHeadingsWith{\bfheadings}
%\FormatHeadingsWith{\uppercase}
%\FormatHeadingsWith{\underline}
\newcommand\upun[1]{\uppercase{\underline{\underline{#1}}}}
\FormatHeadingsWith\upun

\newcommand\itheadings[1]{\textit{#1}}
\FormatHeadingsWith{\itheadings}

% pour avoir un trait sous l'en-tete:
\setlength{\HeadRuleWidth}{0.4pt}
