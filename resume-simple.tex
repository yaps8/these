Cette thèse porte sur l'analyse et la détection de programmes malveillants présentant certaines techniques de protection.
Contrairement à un logiciel classique, ces programmes modifient leur code au fur et à mesure de leur exécution.
Nous proposons une analyse combinant exécution et observation passive du programme permettant de reconstruire un graphe représentant toutes les actions possibles du programme.

Nous étudions également un détecteur de programmes malveillants, fonctionnant par comparaison de ces graphes reconstruits lors de la phase d'analyse : cet antivirus compare le graphe d'un programme à analyser aux graphes de programmes connus pour être malveillants et en déduit une classification.
Nous proposons une formalisation de ce problème de comparaison de graphes ainsi que des techniques permettant de le résoudre efficacement.