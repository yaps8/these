 
\documentclass{beamer}

\usepackage[utf8x]{inputenc}
\usepackage{default}

\usepackage{graphicx} % pour insérer des images
%\usepackage{graphics} % pour insérer des images vectorielles (ps, eps)
\usepackage{epstopdf}

% les figure imbriquées
\usepackage{epsfig}
\usepackage{subfigure}


% \newtheorem{de}{Définition}
\setbeamertemplate{navigation symbols}{}
\newtheorem{prop}{Property}

% \usetheme{Warsaw}
 \usetheme{Frankfurt}
% \setbeamertemplate{footnote}{%
%   \hangpara{2em}{1}%
%   \makebox[2em][l]{\insertfootnotemark}\footnotesize\insertfootnotetext\par%
% }


\mode<presentation>{
\setbeamertemplate{footline}[frame number] %les numéros de page
\setbeamertemplate{bibliography item}[text]
} 
\title[Recognition of binary patterns by Morphological Analysis]{Recognition of binary patterns by Morphological Analysis}
\author{\textbf{Aurélien Thierry} (aurelien.thierry@inria.fr), Guillaume Bonfante, Joan Calvet, Jean-Yves Marion, Fabrice Sabatier}
%\institute{INRIA, Loria}
\date{Recon 2012-06-16}
% \titlegraphic{\includegraphics[width=0.1\textwidth]{../images/loria.jpg}}
% \titlegraphic{\includegraphics[width=0.1\textwidth]{../images/logo_inria.jpg}}


\begin{document}
\makeatletter
  \@ifundefined{inserttotalframenumbernew}{
    \gdef\inserttotalframenumbernew{1}
  }{}
  \gdef\inserttotalframenumber{\inserttotalframenumbernew}
\makeatother

\section{Introduction}
\begin{frame}
\titlepage
\begin{figure}[ht]
\begin{center}
  \subfigure{
\label{fig:CFGConstr}
\epsfig{figure=TUL/Inria.png,height=1.2cm}}\quad
  \subfigure{
\label{fig:CFGNorm}
\epsfig{figure=TUL/tulloria.pdf,height=1.3cm}}\
\subfigure{
\label{fig:CFGConstr}
\epsfig{figure=TUL/tulul.pdf,height=1.0cm}}\quad
\end{center}
\label{fig:CFGConstrNorm}
% \caption{Virut.a without (99 nodes) and with (41 nodes) reduction}
\end{figure}
\end{frame}


 \begin{frame}{Introduction}

\begin{itemize}
%  \item Identify librairies used by binaries
 \item Binary analysis
%  \item Waledac uses cryptography : which algorithms / librairies ?
%  \item Duqu shares code with Stuxnet ? Which parts ?
 \begin{figure}
\end{figure}
 \item<2-> Identify librairies that do not need to be reversed
\end{itemize}
% \pause
\only<3->{
Our approach :
\begin{itemize}
 \item Control flow graph comparison
 \item Import results in IDA
\end{itemize}
}
\end{frame}



\begin{frame}{Duqu / Stuxnet : summary}
\begin{itemize}
 \item From the decrypted and unpacked DLLs from Stuxnet, we are able to automatically find code shared with Duqu
 \item Before reversing, we identify standard (msvcr80.dll) subroutines
 \item With IDA, we can identify and browse matching subroutines
\end{itemize}
\end{frame}

\section{Conclusion}

\begin{frame}{Conclusion}
\begin{itemize}
\item Identify used librairies
\item Show code similarities
\item IDA UI for browsing matched code
\end{itemize}
% Still to be done :
\pause
\begin{itemize}
\item Thank you
\item Any question ? (aurelien.thierry@inria.fr)
% \item Loop detection and comparison for dynamic analysis
% \item Better reductions to detect through obfuscation
% \item Smoother UI for the IDA plugin
\end{itemize}
\end{frame}



\makeatletter
  \immediate\write\@mainaux{\string\gdef\string\inserttotalframenumbernew{\insertframenumber}}
\makeatother
\appendix


% \begin{frame}[allowframebreaks]
% \section*{}
% {
% \frametitle{Références}
% \bibliography{../rapportbib}
% \bibliographystyle{alpha}
% }
% \end{frame}

% % suppléments :
% \begin{frame}{More}
% \end{frame}




\end{document}
